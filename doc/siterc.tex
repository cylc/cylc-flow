
\section{Site/User Config File Reference}
\label{SiteRCReference}

\lstset{language=bash}

This appendix defines all cylc site and user config items - settings
that apply globally to all suites. If a site file exists 
(\lstinline=$CYLC_DIR/conf/siterc/site.rc=) any settings it will
override the system defaults. If a user file exists 
(\lstinline=$HOME/.cylc/user.rc=) any settings in it will  
override the site or system defaults.

To generate an initial site or user config file:
\begin{lstlisting}
$ cylc get-global-config --print > site.rc # or user.rc
\end{lstlisting}
then comment out or delete any items that do not need to be overridden, 
and copy the file to the appropriate location.

{\em Note that Jinja2 expressions can be embedded in site and user
config files to programmatically generate the final result parsed by
cylc.}  Use of Jinja2 in suite definitions is documented in
Section~\ref{Jinja2}.

\subsection{Top Level Items}

\subsubsection{temporary directory}

A temporary directory is needed by a few cylc commands, and is cleaned
automatically on exit. Leave unset for the default (usually
\lstinline=$TMPDIR=).

\begin{myitemize}
\item {\em type:} string (directory path)
\item {\em default:} (none)
\item {\em example:} \lstinline@temporary directory = /tmp/$USER/cylc@
\end{myitemize}

\subsubsection{state dump rolling archive length}

A rolling archive of suite state dumps is maintained under the suite run
directory, and is used for restarts; this item determines the number of
previous states retained. The most recent saved state file is called
\lstinline=state=. Sucessively older files have increasing integer
values appended, starting from $1$.

\begin{myitemize}
\item {\em type:} integer
\item {\em default:} 10
\item {\em example:} \lstinline@state dump rolling archive length = 20@
\end{myitemize}

\subsubsection{disable interactive command prompts}

Commands that intervene in running suites can be made to ask for
confirmation before acting. Some find this annoying and ineffective as a
safety measure, however, so command prompts are disabled by default.

\begin{myitemize}
\item {\em type:} boolean
\item {\em default:} True
\end{myitemize}

\subsubsection{enable run directory housekeeping}

The suite run directory tree is created anew with every suite start
(not restart) but output from the most recent previous runs can be 
retained in a rolling archive. Set length to 0 to keep no backups.
{\bf This is incompatible with current rose suite housekeeping} so it is
disabled by default, in which case (if you are not using Rose) new suite
run files will overwrite existing ones in the same run directory tree,
which can in some cases result in incorrect task polling results.

\begin{myitemize}
\item {\em type:} boolean
\item {\em default:} False
\end{myitemize}

\subsubsection{run directory rolling archive length}

The number of old run directory trees to retain if run directory
housekeeping is enabled.
\begin{myitemize}
\item {\em type:} integer
\item {\em default:} 2
\end{myitemize}

\subsubsection{execution polling intervals}

Cylc can poll running jobs to catch problems that prevent task messages
from being sent back to the suite, such as hard job kills, network
outages, or unplanned task host shutdown. Routine polling is done only 
for the polling {\em task communication method} (below) unless 
suite-specific polling is configured in the suite definition.
A list of interval values can be specified, with the last value used
repeatedly until the task is finished - this allows more frequent
polling near the beginning and end of the anticipated task run time.
Multipliers can be used as shorthand as in the example below.

\begin{myitemize}
\item {\em type:} list of float minutes with optional multipliers
\item {\em default:} 1.0
\item {\em example:} \lstinline@execution polling intervals = 5*1.0, 10*5.0, 5*1.0@
\end{myitemize}


\subsubsection{submission polling intervals}

Cylc can also poll submitted jobs to catch problems that prevent the
submitted job from executing at all, such as deletion from an external 
batch scheduler queue. Routine polling is done only for the polling {\em
task communication method} (below) unless suite-specific polling
is configured in the suite definition. A list of interval
values can be specified as for execution polling (above) but a single
value is probably sufficient for job submission polling. 

\begin{myitemize}
\item {\em type:} list of float minutes with optional multipliers
\item {\em default:} 1.0 (single value)
\item {\em example:} (see the execution polling example above)
\end{myitemize}

\subsection{[task messaging]}

This section contains configuration items that affect task-to-suite
communications.

\subsubsection[retry interval in seconds]{[task messaging] $\rightarrow$ retry interval in seconds}

If a send fails, the messaging code will retry after a configured
delay interval.

\begin{myitemize}
\item {\em type:} float
\item {\em minimum:} 1.0
\item {\em default:} 5.0
\end{myitemize}

\subsubsection[maximum number of tries]{[task messaging] $\rightarrow$ maximum number of tries}

If successive sends fail, the messaging code will give up after a
configured number of tries. 

\begin{myitemize}
\item {\em type:} integer
\item {\em minimum:} 1
\item {\em default:} 7
\end{myitemize}

\subsubsection[connection timeout in seconds]{[task messaging] $\rightarrow$ connection timeout in seconds}

This is the same as the \lstinline=--pyro-timeout= option in cylc
commands. Without a timeout Pyro connections to unresponsive
suites can hang indefinitely (suites suspended with Ctrl-Z for instance).

\begin{myitemize}
\item {\em type:} float
\item {\em minimum:} 1.0
\item {\em default:} 30.0
\end{myitemize}

\subsection{[suite logging]}

The suite event log, held under the suite run directory, is maintained
as a rolling archive. Logs are rolled over (backed up and started anew)
when they reach a configurable limit size.

\subsubsection[roll over at start-up]{[suite logging] $\rightarrow$ roll over at start-up}

If true, a new suite log will be started for a new suite run.

\begin{myitemize}
\item {\em type:} boolean
\item {\em default:} True
\end{myitemize}

\subsubsection[rolling archive length]{[suite logging] $\rightarrow$ rolling archive length}

How many rolled logs to retain in the archive.

\begin{myitemize}
\item {\em type:} integer
\item {\em minimum:} 1
\item {\em default:} 5
\end{myitemize}

\subsubsection[maximum size in bytes]{[suite logging] $\rightarrow$ maximum size in bytes}

Suite event logs are rolled over when they reach this file size. 

\begin{myitemize}
\item {\em type:} integer
\item {\em default:} 1000000 
\end{myitemize}

\subsection{[documentation]}

Documentation locations for the \lstinline=cylc doc= command and gcylc
Help menus. 

\subsubsection[{[[}files{]]}]{[documentation] $\rightarrow$ [files]}

File locations of documentation held locally on the cylc host server.

\paragraph[html index]{[documentation] $\rightarrow$ [files] $\rightarrow$ html index }

File location of the main cylc documentation index.
\begin{myitemize}
\item {\em type:} string
\item {\em default:} \lstinline=$CYLC_DIR/doc/index.html=
\end{myitemize}

\paragraph[pdf user guide]{[documentation] $\rightarrow$ [files] $\rightarrow$ pdf user guide }

File location of the cylc User Guide, PDF version.
\begin{myitemize}
\item {\em type:} string
\item {\em default:} \lstinline=$CYLC_DIR/doc/cug-pdf.pdf=
\end{myitemize}

\paragraph[multi-page html user guide]{[documentation] $\rightarrow$ [files] $\rightarrow$ multi-page html user guide }

File location of the cylc User Guide, multi-page HTML version.
\begin{myitemize}
\item {\em type:} string
\item {\em default:} \lstinline=$CYLC_DIR/doc/html/multi/cug-html.html=
\end{myitemize}

\paragraph[single-page html user guide]{[documentation] $\rightarrow$ [files] $\rightarrow$ single-page html user guide }

File location of the cylc User Guide, single-page HTML version.
\begin{myitemize}
\item {\em type:} string
\item {\em default:} \lstinline=$CYLC_DIR/doc/html/single/cug-html.html=
\end{myitemize}

\subsubsection[{[[}urls{]]}]{[documentation] $\rightarrow$ [urls]}

Online documentation URLs.

\paragraph[internet homepage]{[documentation] $\rightarrow$ [urls] $\rightarrow$ internet homepage }

URL of the cylc internet homepage, with links to documentation for the
latest official release.

\begin{myitemize}
\item {\em type:} string
\item {\em default:} http://cylc.github.com/cylc/
\end{myitemize}

\paragraph[local index]{[documentation] $\rightarrow$ [urls] $\rightarrow$ local index}

Local intranet URL of the main cylc documentation index.

\begin{myitemize}
\item {\em type:} string
\item {\em default:} (none)
\end{myitemize}

\subsection{[document viewers]}

PDF and HTML viewers can be launched by cylc to view the documentation.

\subsubsection[pdf]{[document viewers] $\rightarrow$ pdf}

Your preferred PDF viewer program.

\begin{myitemize}
\item {\em type:} string
\item {\em default:} evince
\end{myitemize}

\subsubsection[html]{[document viewers] $\rightarrow$ html}

Your preferred web browser.

\begin{myitemize}
\item {\em type:} string
\item {\em default:} firefox
\end{myitemize}

\subsection{[editors]}

Choose your favourite text editor for editing suite definitions.

\subsubsection[terminal]{[editors] $\rightarrow$ terminal}

The editor to be invoked by the cylc command line interface.

\begin{myitemize}
\item {\em type:} string
\item {\em default:} \lstinline=vim=
\item {\em examples:}
    \begin{myitemize}
            \item \lstinline@terminal = emacs -nw@ (emacs non-GUI)
            \item \lstinline@terminal = emacs@ (emacs GUI)
            \item \lstinline@terminal = gvim -f@ (vim GUI)
    \end{myitemize}
\end{myitemize}

\subsubsection[gui]{[editors] $\rightarrow$ gui}

The editor to be invoked by the cylc GUI.

\begin{myitemize}
\item {\em type:} string
\item {\em default:} \lstinline=gvim -f=
\item {\em examples:}
    \begin{myitemize}
            \item \lstinline@gui = emacs@
            \item \lstinline@gui = xterm -e vim@
    \end{myitemize}
\end{myitemize}


\subsection{[Pyro]}

Pyro is the RPC layer used for network communication between cylc
clients (suite-connecting commands and guis) servers (running suites).
Each suite listens on a dedicated network port, binding on the first
available starting at the configured base port.

\subsubsection[base port]{[Pyro] $\rightarrow$ base port }

The first port that cylc is allowed to use.

\begin{myitemize}
\item {\em type:} integer
\item {\em default:} 7766
\end{myitemize}

\subsubsection[maximum number of ports]{[Pyro] $\rightarrow$ maximum number of ports}

This determines the maximum number of suites that can run at once on the
suite host.

\begin{myitemize}
\item {\em type:} integer
\item {\em default:} 100
\end{myitemize}

\subsubsection[ports directory]{[Pyro] $\rightarrow$ ports directory}

Each suite stores its port number, by suite name, under this directory.

\begin{myitemize}
\item {\em type:} string (directory path)
\item {\em default:} \lstinline@$HOME/.cylc/ports/"@
\end{myitemize}

\subsection{[hosts]}

The [hosts] section configures some important host-specific settings for
the suite host (`localhost') and remote task hosts. Note that {\em
remote task behaviour is determined by the site/user config file on the
suite host, not on the task host}. Suites can specify task hosts that 
are not listed here, in which case local settings will be assumed,
with the local home directory path, if present, replaced by
\lstinline=$HOME= in items that configure directory locations.

\subsubsection[{[[}HOST{]]}]{[hosts] $\rightarrow$ HOST}

The default task HOST is the suite host, {\bf localhost}, with default 
values as listed below. Use an explicit \lstinline=[hosts][[localhost]]=
section if you need to override the defaults. Localhost settings are
then also used as defaults for other hosts, with the local home
directory path replaced as described above. This applies to items
omitted from an explicit host section, and to hosts that are not listed
at all in the site/user config files.  Explicit host sections are only
needed if the automatically modified local defaults are not sufficient.

Host section headings can also be {\em regular expressions} to match 
multiple hostnames.  Note that the general regular expression wildcard
is `\lstinline=.*=' (zero or more of any character), not 
`\lstinline=*='.
Hostname matching regular expressions are used as-is in the Python
\lstinline=re.match()= function. As such they match from the beginning 
of the hostname string (as specified in the suite definition) and they
do not have to match through to the end of the string (use the
string-end matching character `\lstinline=$=' in the expression to 
force this).

A hierachy of host match expressions from specific to general can be 
used because config items are processed in the order specified in the
file.

\begin{myitemize}
\item {\em type:} string (hostname or regular expression)
\item {\em examples:}
\begin{myitemize}
    \item \lstinline@server1.niwa.co.nz@ - explicit host name
    \item  \lstinline@server\d.niwa.co.nz@ - regular expression
\end{myitemize}
\end{myitemize}

\paragraph[run directory]{[hosts] $\rightarrow$ HOST $\rightarrow$ run directory }

The top level of the directory tree that holds suite-specific output logs,
state dump files, run database, etc.

\begin{myitemize}
\item {\em type:} string (directory path)
\item {\em default:} \lstinline=$HOME/cylc-run=
\end{myitemize}

\paragraph[work directory]{[hosts] $\rightarrow$ HOST $\rightarrow$ work directory }

The top level for suite work and share directories.

\begin{myitemize}
\item {\em type:} string (directory path)
\item {\em localhost default:} \lstinline@$HOME/cylc-run@
\end{myitemize}


\paragraph[task communication method]{[hosts] $\rightarrow$ HOST $\rightarrow$ task communication method }

The means by which task progress messages are reported back to the running suite.
See above for default polling intervals for the poll method.

\begin{myitemize}
\item {\em type:} string (must be one of the following three options)
\item {\em options:}
    \begin{myitemize}
    \item {\bf pyro} - direct client-server RPC via network ports
    \item {\bf ssh} - use ssh to re-invoke the Pyro messaging commands on the suite server
    \item {\bf poll} - the suite polls for the status of tasks (no task messaging)
  \end{myitemize}
\item {\em localhost default:} pyro
\end{myitemize}

\paragraph[remote shell template]{[hosts] $\rightarrow$ HOST $\rightarrow$ remote shell template }

A string template, containing \lstinline@%s@ as a placeholder for the host name,
for the command used to invoke commands on this host.
This is not used on the suite host unless you run local tasks under
another user account.

\begin{myitemize}
\item {\em type:} string
\item {\em localhost default:} \lstinline@ssh -oBatchMode=yes %s@ )
\end{myitemize}

\paragraph[use login shell]{[hosts] $\rightarrow$ HOST $\rightarrow$ use login shell }

Whether to use a login shell or not for remote command invocation. By
default cylc runs remote ssh commands using a login shell,
\begin{lstlisting}
  ssh user@host 'bash --login cylc ...'
\end{lstlisting}
which will source \lstinline=/etc/profile= and 
\lstinline=~/.profile= to set up the user environment.  However, for
security reasons some institutions do not allow unattended commands to
start login shells, so you can turn off this behaviour to get,
\begin{lstlisting}
  ssh user@host 'cylc ...'
\end{lstlisting}
which will use the default shell on the remote machine,
sourcing \lstinline=~/.bashrc= (or \lstinline=~/.cshrc=) to set up the
environment.
In either case \lstinline=$PATH= on the remote machine should include
\lstinline@$CYLC_DIR/bin@ in order for the remote cylc program to be found.

{\em NOTE: this setting does not currently apply to job submission
commands (which execute on the suite host to submit remote tasks). }

\begin{myitemize}
\item {\em type:} boolean
\item {\em localhost default:} true
\end{myitemize}

\subsection{[suite host self-identification] }

The suite host's identity must be determined locally by cylc and passed
to running tasks (via \lstinline@$CYLC_SUITE_HOST@) so that task messages
can target the right suite on the right host.

%(TO DO: is it conceivable that different remote task hosts at the same
%site might see the suite host differently? If so we would need to be
%able to override the target in suite definitions.)

\subsubsection[method]{[suite host self-identification] $\rightarrow$ method }

This item determines how cylc finds the identity of the suite host. For
the default {\em name} method cylc asks the suite host for its host
name. This should resolve on remote task hosts to the IP address of the
suite host; if it doesn't, adjust network settings or use one of the
other methods. For the {\em address} method, cylc attempts to use a
special external ``target address'' to determine the IP address of the
suite host as seen by remote task hosts (in-source documentation in
\lstinline=$CYLC_DIR/lib/cylc/suite_host.py= explains how this works).
And finally, as a last resort, you can choose the {\em hardwired} method
and manually specify the host name or IP address of the suite host.

\begin{myitemize}
\item {\em type:} string
\item {\em options:} 
\begin{myitemize}
    \item name - self-identified host name
    \item address - automatically determined IP address (requires {\em target}, below)
    \item hardwired - manually specified host name or IP address (requires {\em host}, below)
\end{myitemize}
\item {\em default:} name
\end{myitemize}

\subsubsection[target]{[suite host self-identification] $\rightarrow$ target }

This item is required for the {\em address} self-identification method.
If your suite host sees the internet, a common address such as
\lstinline@google.com@ will do; otherwise choose a host visible on your
intranet.
\begin{myitemize}
\item {\em type:} string (an inter- or intranet URL visible from the suite host)
\item {\em default:} \lstinline@google.com@
\end{myitemize}


\subsubsection[host]{[suite host self-identification] $\rightarrow$ host }

Use this item to explicitly set the name or IP address of the suite host  
if you have to use the {\em hardwired} self-identification method.
\begin{myitemize}
\item {\em type:} string (host name or IP address)
\item {\em default:} (none)
\end{myitemize}

\subsection{[suite host scanning]}

Utilities such as \lstinline=cylc gsummary= need to scan hosts for
running suites.

\subsubsection[hosts]{[suite host scanning] $\rightarrow$ hosts }

A list of hosts to scan for running suites.
\begin{myitemize}
\item {\em type:} comma-separated list of host names or IP addresses.
\item {\em default:} localhost
\end{myitemize}

