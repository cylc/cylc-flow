\section{Site.rc Reference}
\label{SiteRCReference}

\lstset{language=bash}

*** NOT USED YET - FROM SUITE DEF ITEMS CUT FROM OLD SUITERC.TEX - To Be Completed ***

\subsection[run directory]{suite run directory}

Cylc writes the following files to a special ``run'' directory:

\begin{myitemize}
    \item suite event log, and stdout and stderr logs, in \lstinline=SUITE/log/suite/=
    \item task stdout and stderr logs, in \lstinline=SUITE/log/job/= 
    \item suite state dump files used for restarts, in \lstinline=SUITE/state/=)
\end{myitemize}
Where \lstinline=SUITE= is the suite name.

\begin{myitemize}
    \item {\em type:} string (directory path, may contain environment variables)
\end{myitemize}

\subsection[state dump rolling archive length]{[cylc] $\rightarrow$ state dump rolling archive length}

This is the length, in number of changes, of the automatic rolling
archive of state dump files that allows you to restart a suite from a
previous state.  Every time a task changes state cylc updates the state
dump and rolls previous states back one on the archive. You'll probably
only ever need the latest (most recent) state dump, which is
automatically used in a restart, but any previous state still in the
archive can be used.  Additionally, special labeled state dumps are
written prior to actioning any suite intervention, and their filenames
are logged by cylc.

\begin{myitemize}
    \item {\em type:} integer ($\geq 1$)
    \item {\em default:} $10$
\end{myitemize}


\subsection{[logging]}

\subsubsection[roll over at start-up]{[suite logging] $\rightarrow$ roll over at start-up}

Suite logs roll over (start anew) automatically when they reach the
configured maximum size, and whenever the suite is started or restarted.

\begin{myitemize}
    \item {\em type:} boolean
    \item {\em default:} True
\end{myitemize}


\subparagraph[cylc directory]{[runtime] $\rightarrow$ [[\_\_NAME\_\_]] $\rightarrow$ [[[remote]]] $\rightarrow$ cylc directory}

The path to the remote cylc installation, required if cylc
is not in the default search path on the remote host.

\begin{myitemize}
\item {\em type:} string (a valid directory path on the remote host)
\item {\em default:} (none)
\end{myitemize}

\subparagraph[remote shell template]{[runtime] $\rightarrow$ [[\_\_NAME\_\_]] $\rightarrow$ [[[remote]]] $\rightarrow$ remote shell template }

A template for the remote shell command for a submitting a remote task.
The template's first \%s will be substituted by the remote user@host.

\begin{myitemize}
\item {\em type:} string (a string template)
\item {\em root default:} \lstinline@ssh -oBatchMode=yes %s@
\end{myitemize}

\subparagraph[log directory]{[runtime] $\rightarrow$ [[\_\_NAME\_\_]] $\rightarrow$ [[[remote]]] $\rightarrow$ log directory }

This log directory is used for the stdout and stderr logs of remote
tasks. The directory will be created on the fly if necessary. If not
specified, the local job submission log path will be used
(see~\ref{LocalLog}) with the
suite owner's home directory path, if present, replaced by
\lstinline='$HOME'= for interpretation on the remote host. The stdout
and stderr log file names are the same as for local tasks, and are
recorded by the task proxies for access via gcylc. Suite identity
variables can be used in the path, but {\em not task identity variables}
such as \lstinline=$CYLC_TASK_NAME= and \lstinline=$CYLC_TASK_CYCLE_TIME=, 
because the log directory is created before the task runs.

\begin{myitemize}
\item {\em type:} string (a valid directory path on the remote host)
\item {\em default:} (local log path with \lstinline=$HOME= replaced)
\end{myitemize}
 
NOTE: work and share directories are now configured by a single ``workspace directory''.

\subparagraph[work directory]{[runtime] $\rightarrow$ [[\_\_NAME\_\_]] $\rightarrow$ [[[remote]]] $\rightarrow$ work directory}

Use this item if you need to override the local task work directory 
(see~\ref{LocalWork}). If omitted, the local directory will be used with
the suite owner's home directory path, if present,
replaced by \lstinline='$HOME'= for interpretation on 
the remote host.

\begin{myitemize}
\item {\em type:} string (directory path, may contain environment variables)
\item {\em default:} (local task work path with \lstinline=$HOME= replaced)
\end{myitemize}

\subparagraph[share directory]{[runtime] $\rightarrow$ [[\_\_NAME\_\_]] $\rightarrow$ [[[remote]]] $\rightarrow$ share directory}

Use this item if you need to override the local share directory 
(see~\ref{LocalShare}). If omitted, the local directory will be used
with the suite owner's home directory path, if present, replaced by
\lstinline='$HOME'= for interpretation on the remote host.

\begin{myitemize}
\item {\em type:} string (directory path, may contain environment variables)
\item {\em default:} (local task share path with \lstinline=$HOME= replaced)
\end{myitemize}

\subparagraph[ssh messaging]{[runtime] $\rightarrow$ [[\_\_NAME\_\_]] $\rightarrow$ [[[remote]]] $\rightarrow$ ssh messaging}

If your network configuration or firewall blocks the TCP/IP sockets
required for remote tasks to communicate with their parent suite
you can tell cylc to use passwordless ssh (from remote host to suite
host) instead, to invoke local messaging commands on the suite host.

\begin{myitemize}
\item {\em type:} boolean
\item {\em default:} False
\end{myitemize}

This item affects the behaviour of the cylc messaging commands by means
of the task execution environment, so no special cylc configuration is
required on the remote host itself. Eventually it may be moved to a site
and host configuration file (yet to be implemented) because it is host-
rather than task-specific. For the moment though you can still set it
just once in a namespace inherited by all tasks on the affected host.

Note that you can use a remote section with this item in it even for
tasks that are local as far as cylc is concerned, but which end up
running on the affected remote host due to the action of the local batch
queueing system or resource manager.


\subparagraph[ssh messaging]{[runtime] $\rightarrow$ [[\_\_NAME\_\_]] $\rightarrow$ [[[remote]]] $\rightarrow$ use login shell}

By default Cylc will submit remote ssh commands using a login shell. For
security reasons some institutions do not allow unattended commands to start
login shells, setting this item to false will disable that behaviour.

When this option is set to True Cylc will start a Bash login shell to run
remote ssh commands, e.g. \lstinline=ssh user@host 'bash --login cylc task'=
which will source the files \lstinline=/etc/profile= and \lstinline=~/.profile=
in order to set up the user environment. Without the login option Cylc will be
run directly by ssh, e.g. \lstinline=ssh user@host 'cylc task'= which will use
the default shell on the remote machine. In this case the environment will be
set up by sourcing the files \lstinline=~/.bashrc= or \lstinline=~/.cshrc=,
depending on the shell type of the remote machine.

In either case the PATH environment variable on the remote machine should
include \lstinline=$CYLC_DIR/bin= in order for the Cylc executable to be
found.

\begin{myitemize}
\item {\em type:} boolean
\item {\em default:} True
\end{myitemize}

