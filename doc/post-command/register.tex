
Registration creates a mapping between a name and a system definition
directory, and stores this in \lstinline=$HOME/.cylc/registrations=.
This allows users to access particular systems without having to
constantly type in the system definition directory path. 

The registered name under which a system is running is made available to
system tasks via their execution environment. If all system tasks are
configured to use the registered name in all important input and output
directories, then it is possible to run multiple instances of a system
at the same time without interference between them.

The reason that a running system can only be accessed, by cylc commands,
via the registered name under which it was started up, is that the name
is used, along with username, to register (with Pyro now!) system
objects in the Pyro nameserver.


