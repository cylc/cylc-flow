Registration serves two purposes: (1) cylc commands that need to access
system-specific modules can do so without the user constantly having to
type the full system definition directory path on the command line; and
(2) the registered name is used along with your username to distinguish
between different systems using the Pyro nameserver at the same time (so
different users can register the same system under the same name). 

Registration is only required if you want to start scheduling a system
(\lstinline=cylc start= or \lstinline=cylc restart=), run a single task
from a system (\lstinline=cylc run-task=), or print system-specific
descriptive information (\lstinline=system-info=), because these
commands need to access system-specific code modules.

Other cylc commands interact with a running target system via the Pyro
nameserver; for these you need to know the registered name of the
system, and the username under which it is running (use 
\lstinline=cylc list= to discover which systems are running), but you
don't need to have registered it yourself. 

The userguide example system uses the registered system name in all file
I/O operations, so you can run multiple instances of it at the same time
if each instance is registered under a different name.  This may be
useful, at least in relatively small systems that are not too
resource-constrained, to run several historical case studies at once
over different time periods, for instance, or to run a case study at the
same time as the real time operation using the exact operational system
(which guarantees the two systems really are identical). The alternative
to this would be to (i) copy the system definition directory to create
an entirely new system based on the original, and (ii) make sure the
task definitions also invoke different scripts/programs to run the real
tasks.

Note that the cylc lockserver prevents multiple instances of a system
from running at the same time, even under different names or different
users, unless the system config file says the system can handle this.
