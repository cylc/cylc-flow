\label{help}
\begin{lstlisting}
Cylc ("silk") is a workflow engine for orchestrating complex
*suites* of inter-dependent distributed cycling (repeating) tasks, as well as
ordinary non-cycling workflows.
For detailed documentation see the Cylc User Guide (cylc doc --help).

Version UNKNOWN

The graphical user interface for cylc is "gcylc" (a.k.a. "cylc gui").

USAGE:
  % cylc -V,--version,version           # print cylc version
  % cylc version --long                 # print cylc version and path
  % cylc help,--help,-h,?               # print this help page

  % cylc help CATEGORY                  # print help by category
  % cylc CATEGORY help                  # (ditto)
  % cylc help [CATEGORY] COMMAND        # print command help
  % cylc [CATEGORY] COMMAND --help      # (ditto)
  % cylc COMMAND --help                 # (ditto)

  % cylc COMMAND [options] SUITE [arguments]
  % cylc COMMAND [options] SUITE TASK [arguments]

Commands can be abbreviated as long as there is no ambiguity in
the abbreviated command:

  % cylc trigger SUITE TASK             # trigger TASK in SUITE
  % cylc trig SUITE TASK                # ditto
  % cylc tr SUITE TASK                  # ditto

  % cylc get                            # Error: ambiguous command

TASK IDENTIFICATION IN CYLC SUITES
  Tasks are identified by NAME.CYCLE_POINT where POINT is either a
  date-time or an integer.
  Date-time cycle points are in an ISO 8601 date-time format, typically
  CCYYMMDDThhmm followed by a time zone - e.g. 20101225T0600Z.
  Integer cycle points (including those for one-off suites) are integers
  - just '1' for one-off suites.

HOW TO DRILL DOWN TO COMMAND USAGE HELP:
  % cylc help           # list all available categories (this page)
  % cylc help prep      # list commands in category 'preparation'
  % cylc help prep edit # command usage help for 'cylc [prep] edit'

Command CATEGORIES:
  control ....... Suite start up, monitoring, and control.
  information ... Interrogate suite definitions and running suites.
  all ........... The complete command set.
  task .......... The task messaging interface.
  license|GPL ... Software licensing information (GPL v3.0).
  admin ......... Cylc installation, testing, and example suites.
  preparation ... Suite editing, validation, visualization, etc.
  hook .......... Suite and task event hook scripts.
  discovery ..... Detect running suites.
  utility ....... Cycle arithmetic and templating, etc.
\end{lstlisting}
\subsection{Command Categories}
\subsubsection{admin}
\label{admin}
\begin{lstlisting}
CATEGORY: admin - Cylc installation, testing, and example suites.

HELP: cylc [admin] COMMAND help,--help
  You can abbreviate admin and COMMAND.
  The category admin may be omitted.

COMMANDS:
  check-software .... Check required software is installed
  import-examples ... Import example suites your suite run directory
  profile-battery ... Run a battery of profiling tests
  test-battery ...... Run a battery of self-diagnosing test suites
  upgrade-run-dir ... Upgrade a pre-cylc-6 suite run directory
\end{lstlisting}
\subsubsection{all}
\label{all}
\begin{lstlisting}
CATEGORY: all - The complete command set.

HELP: cylc [all] COMMAND help,--help
  You can abbreviate all and COMMAND.
  The category all may be omitted.

COMMANDS:
  5to6 ........................................ Improve the cylc 6 compatibility of a cylc 5 suite file
  broadcast|bcast ............................. Change suite [runtime] settings on the fly
  cat-log|log ................................. Print various suite and task log files
  cat-state ................................... Print the state of tasks from the state dump
  check-software .............................. Check required software is installed
  check-triggering ............................ A suite shutdown event hook for cylc testing
  check-versions .............................. Compare cylc versions on task host accounts
  checkpoint .................................. Tell suite to checkpoint its current state
  client ...................................... (Internal) Invoke HTTP(S) client, expect JSON input
  conditions .................................. Print the GNU General Public License v3.0
  cycle-point|cyclepoint|datetime|cycletime ... Cycle point arithmetic and filename templating
  diff|compare ................................ Compare two suite definitions and print differences
  documentation|browse ........................ Display cylc documentation (User Guide etc.)
  dump ........................................ Print the state of tasks in a running suite
  edit ........................................ Edit suite definitions, optionally inlined
  email-suite ................................. A suite event hook script that sends email alerts
  email-task .................................. A task event hook script that sends email alerts
  ext-trigger|external-trigger ................ Report an external trigger event to a suite
  function-run ................................ (Internal) Run a function in the process pool
  get-directory ............................... Retrieve suite source directory paths
  get-gui-config .............................. Print gcylc configuration items
  get-host-metrics ............................ Print localhost metric data
  get-site-config|get-global-config ........... Print site/user configuration items
  get-suite-config|get-config ................. Print suite configuration items
  get-suite-contact|get-contact ............... Print contact information of a suite server program
  get-suite-version|get-cylc-version .......... Print cylc version of a suite server program
  gpanel ...................................... Internal interface for GNOME 2 panel applet
  graph ....................................... Plot suite dependency graphs and runtime hierarchies
  graph-diff .................................. Compare two suite dependencies or runtime hierarchies
  gscan|gsummary .............................. Scan GUI for monitoring multiple suites
  gui ......................................... (a.k.a. gcylc) cylc GUI for suite control etc.
  hold ........................................ Hold (pause) suites or individual tasks
  import-examples ............................. Import example suites your suite run directory
  insert ...................................... Insert tasks into a running suite
  jobs-kill ................................... (Internal) Kill task jobs
  jobs-poll ................................... (Internal) Retrieve status for task jobs
  jobs-submit ................................. (Internal) Submit task jobs
  jobscript ................................... Generate a task job script and print it to stdout
  kill ........................................ Kill submitted or running tasks
  list|ls ..................................... List suite tasks and family namespaces
  ls-checkpoints .............................. Display task pool etc at given events
  message|task-message ........................ Report task messages
  monitor ..................................... An in-terminal suite monitor (see also gcylc)
  nudge ....................................... Cause the cylc task processing loop to be invoked
  ping ........................................ Check that a suite is running
  poll ........................................ Poll submitted or running tasks
  print ....................................... Print registered suites
  profile-battery ............................. Run a battery of profiling tests
  register .................................... Register a suite for use
  release|unhold .............................. Release (unpause) suites or individual tasks
  reload ...................................... Reload the suite definition at run time
  remote-init ................................. (Internal) Initialise a task remote
  remote-tidy ................................. (Internal) Tidy a task remote
  remove ...................................... Remove tasks from a running suite
  report-timings .............................. Generate a report on task timing data
  reset ....................................... Force one or more tasks to change state
  restart ..................................... Restart a suite from a previous state
  review ...................................... Start/stop ad-hoc Cylc Review web service server.
  run|start ................................... Start a suite at a given cycle point
  scan ........................................ Scan a host for running suites
  scp-transfer ................................ Scp-based file transfer for cylc suites
  search|grep ................................. Search in suite definitions
  set-verbosity ............................... Change a running suite's logging verbosity
  show ........................................ Print task state (prerequisites and outputs etc.)
  spawn ....................................... Force one or more tasks to spawn their successors
  stop|shutdown ............................... Shut down running suites
  submit|single ............................... Run a single task just as its parent suite would
  suite-state ................................. Query the task states in a suite
  test-battery ................................ Run a battery of self-diagnosing test suites
  trigger ..................................... Manually trigger or re-trigger a task
  upgrade-run-dir ............................. Upgrade a pre-cylc-6 suite run directory
  validate .................................... Parse and validate suite definitions
  view ........................................ View suite definitions, inlined and Jinja2 processed
  warranty .................................... Print the GPLv3 disclaimer of warranty
\end{lstlisting}
\subsubsection{control}
\label{control}
\begin{lstlisting}
CATEGORY: control - Suite start up, monitoring, and control.

HELP: cylc [control] COMMAND help,--help
  You can abbreviate control and COMMAND.
  The category control may be omitted.

COMMANDS:
  broadcast|bcast ................ Change suite [runtime] settings on the fly
  checkpoint ..................... Tell suite to checkpoint its current state
  client ......................... (Internal) Invoke HTTP(S) client, expect JSON input
  ext-trigger|external-trigger ... Report an external trigger event to a suite
  gui ............................ (a.k.a. gcylc) cylc GUI for suite control etc.
  hold ........................... Hold (pause) suites or individual tasks
  insert ......................... Insert tasks into a running suite
  kill ........................... Kill submitted or running tasks
  nudge .......................... Cause the cylc task processing loop to be invoked
  poll ........................... Poll submitted or running tasks
  release|unhold ................. Release (unpause) suites or individual tasks
  reload ......................... Reload the suite definition at run time
  remove ......................... Remove tasks from a running suite
  reset .......................... Force one or more tasks to change state
  restart ........................ Restart a suite from a previous state
  run|start ...................... Start a suite at a given cycle point
  set-verbosity .................. Change a running suite's logging verbosity
  spawn .......................... Force one or more tasks to spawn their successors
  stop|shutdown .................. Shut down running suites
  trigger ........................ Manually trigger or re-trigger a task
\end{lstlisting}
\subsubsection{discovery}
\label{discovery}
\begin{lstlisting}
CATEGORY: discovery - Detect running suites.

HELP: cylc [discovery] COMMAND help,--help
  You can abbreviate discovery and COMMAND.
  The category discovery may be omitted.

COMMANDS:
  check-versions ... Compare cylc versions on task host accounts
  ping ............. Check that a suite is running
  scan ............. Scan a host for running suites
\end{lstlisting}
\subsubsection{hook}
\label{hook}
\begin{lstlisting}
CATEGORY: hook - Suite and task event hook scripts.

HELP: cylc [hook] COMMAND help,--help
  You can abbreviate hook and COMMAND.
  The category hook may be omitted.

COMMANDS:
  check-triggering ... A suite shutdown event hook for cylc testing
  email-suite ........ A suite event hook script that sends email alerts
  email-task ......... A task event hook script that sends email alerts
\end{lstlisting}
\subsubsection{information}
\label{information}
\begin{lstlisting}
CATEGORY: information - Interrogate suite definitions and running suites.

HELP: cylc [information] COMMAND help,--help
  You can abbreviate information and COMMAND.
  The category information may be omitted.

COMMANDS:
  cat-log|log .......................... Print various suite and task log files
  cat-state ............................ Print the state of tasks from the state dump
  documentation|browse ................. Display cylc documentation (User Guide etc.)
  dump ................................. Print the state of tasks in a running suite
  get-gui-config ....................... Print gcylc configuration items
  get-host-metrics ..................... Print localhost metric data
  get-site-config|get-global-config .... Print site/user configuration items
  get-suite-config|get-config .......... Print suite configuration items
  get-suite-contact|get-contact ........ Print contact information of a suite server program
  get-suite-version|get-cylc-version ... Print cylc version of a suite server program
  gpanel ............................... Internal interface for GNOME 2 panel applet
  gscan|gsummary ....................... Scan GUI for monitoring multiple suites
  gui|gcylc ............................ (a.k.a. gcylc) cylc GUI for suite control etc.
  list|ls .............................. List suite tasks and family namespaces
  monitor .............................. An in-terminal suite monitor (see also gcylc)
  review ............................... Start/stop ad-hoc Cylc Review web service server.
  show ................................. Print task state (prerequisites and outputs etc.)
\end{lstlisting}
\subsubsection{license}
\label{license}
\begin{lstlisting}
CATEGORY: license|GPL - Software licensing information (GPL v3.0).

HELP: cylc [license|GPL] COMMAND help,--help
  You can abbreviate license|GPL and COMMAND.
  The category license|GPL may be omitted.

COMMANDS:
  conditions ... Print the GNU General Public License v3.0
  warranty ..... Print the GPLv3 disclaimer of warranty
\end{lstlisting}
\subsubsection{preparation}
\label{preparation}
\begin{lstlisting}
CATEGORY: preparation - Suite editing, validation, visualization, etc.

HELP: cylc [preparation] COMMAND help,--help
  You can abbreviate preparation and COMMAND.
  The category preparation may be omitted.

COMMANDS:
  5to6 ............ Improve the cylc 6 compatibility of a cylc 5 suite file
  diff|compare .... Compare two suite definitions and print differences
  edit ............ Edit suite definitions, optionally inlined
  get-directory ... Retrieve suite source directory paths
  graph ........... Plot suite dependency graphs and runtime hierarchies
  graph-diff ...... Compare two suite dependencies or runtime hierarchies
  jobscript ....... Generate a task job script and print it to stdout
  list|ls ......... List suite tasks and family namespaces
  print ........... Print registered suites
  register ........ Register a suite for use
  search|grep ..... Search in suite definitions
  validate ........ Parse and validate suite definitions
  view ............ View suite definitions, inlined and Jinja2 processed
\end{lstlisting}
\subsubsection{task}
\label{task}
\begin{lstlisting}
CATEGORY: task - The task messaging interface.

HELP: cylc [task] COMMAND help,--help
  You can abbreviate task and COMMAND.
  The category task may be omitted.

COMMANDS:
  jobs-kill .............. (Internal) Kill task jobs
  jobs-poll .............. (Internal) Retrieve status for task jobs
  jobs-submit ............ (Internal) Submit task jobs
  message|task-message ... Report task messages
  remote-init ............ (Internal) Initialise a task remote
  remote-tidy ............ (Internal) Tidy a task remote
  submit|single .......... Run a single task just as its parent suite would
\end{lstlisting}
\subsubsection{utility}
\label{utility}
\begin{lstlisting}
CATEGORY: utility - Cycle arithmetic and templating, etc.

HELP: cylc [utility] COMMAND help,--help
  You can abbreviate utility and COMMAND.
  The category utility may be omitted.

COMMANDS:
  cycle-point|cyclepoint|datetime|cycletime ... Cycle point arithmetic and filename templating
  function-run ................................ (Internal) Run a function in the process pool
  ls-checkpoints .............................. Display task pool etc at given events
  report-timings .............................. Generate a report on task timing data
  scp-transfer ................................ Scp-based file transfer for cylc suites
  suite-state ................................. Query the task states in a suite
\end{lstlisting}
\subsection{Commands}
\subsubsection{5to6}
\label{5to6}
\begin{lstlisting}
Usage: cylc [prep] 5to6 FILE

Suggest changes to a cylc 5 suite file to make it more cylc 6 compatible.
This may be a suite.rc file, an include file, or a suite.rc.processed file.

By default, print the changed file to stdout. Lines that have been changed
are marked with '# UPGRADE'. These marker comments are purely for your own
information and should not be included in any changes you make. In
particular, they may break continuation lines.

Lines with '# UPGRADE CHANGE' have been altered.
Lines with '# UPGRADE ... INFO' indicate that manual change is needed.

As of cylc 7, 'cylc validate' will no longer print out automatic dependency
section translations. At cylc 6 versions of cylc, 'cylc validate' will show
start-up/mixed async replacement R1* section(s). The validity of these can
be highly dependent on the initial cycle point choice (e.g. whether it is
T00 or T12).

This command works best for hour-based cycling - it will always convert
e.g. 'foo[T-6]' to 'foo[-PT6H]', even where this is in a monthly or yearly
cycling section graph.

This command is an aid, and is not an auto-upgrader or a substitute for
reading the documentation. The suggested changes must be understood and
checked by hand.

Example usage:

# Print out a file path (FILE) with suggested changes to stdout.
cylc 5to6 FILE

# Replace the file with the suggested changes file.
cylc 5to6 FILE > FILE

# Save a copy of the changed file.
cylc 5to6 FILE > FILE.5to6

# Show the diff of the changed file vs the original file.
diff - <(cylc 5to6 FILE) <FILE

Options:
  -h, --help   Print this help message and exit.
\end{lstlisting}
\subsubsection{broadcast}
\label{broadcast}
\begin{lstlisting}
Usage: cylc [control] broadcast|bcast [OPTIONS] REG

Override [runtime] config in targeted namespaces in a running suite.

Uses for broadcast include making temporary changes to task behaviour,
and task-to-downstream-task communication via environment variables.

A broadcast can target any [runtime] namespace for all cycles or for a
specific cycle.  If a task is affected by specific-cycle and all-cycle
broadcasts at once, the specific takes precedence. If a task is affected
by broadcasts to multiple ancestor namespaces, the result is determined
by normal [runtime] inheritance. In other words, it follows this order:

all:root -> all:FAM -> all:task -> tag:root -> tag:FAM -> tag:task

Broadcasts persist, even across suite restarts, until they expire when
their target cycle point is older than the oldest current in the suite,
or until they are explicitly cancelled with this command.  All-cycle
broadcasts do not expire.

For each task the final effect of all broadcasts to all namespaces is
computed on the fly just prior to job submission.  The --cancel and
--clear options simply cancel (remove) active broadcasts, they do not
act directly on the final task-level result. Consequently, for example,
you cannot broadcast to "all cycles except Tn" with an all-cycle
broadcast followed by a cancel to Tn (there is no direct broadcast to Tn
to cancel); and you cannot broadcast to "all members of FAMILY except
member_n" with a general broadcast to FAMILY followed by a cancel to
member_n (there is no direct broadcast to member_n to cancel).

To broadcast a variable to all tasks (quote items with internal spaces):
  % cylc broadcast -s "[environment]VERSE = the quick brown fox" REG
To do the same with a file:
  % cat >'broadcast.rc' <<'__RC__'
  % [environment]
  %     VERSE = the quick brown fox
  % __RC__
  % cylc broadcast -F 'broadcast.rc' REG
To cancel the same broadcast:
  % cylc broadcast --cancel "[environment]VERSE" REG
If -F FILE was used, the same file can be used to cancel the broadcast:
  % cylc broadcast -G 'broadcast.rc' REG

Use -d/--display to see active broadcasts. Multiple --cancel options or
multiple --set and --set-file options can be used on the same command line.
Multiple --set and --set-file options are cumulative.

The --set-file=FILE option can be used when broadcasting multiple values, or
when the value contains newline or other metacharacters. If FILE is "-", read
from standard input.

Broadcast cannot change [runtime] inheritance.

See also 'cylc reload' - reload a modified suite definition at run time.

Arguments:
   REG               Suite name

Options:
  -h, --help            show this help message and exit
  -p CYCLE_POINT, --point=CYCLE_POINT
                        Target cycle point. More than one can be added.
                        Defaults to '*' with --set and --cancel, and nothing
                        with --clear.
  -n NAME, --namespace=NAME
                        Target namespace. Defaults to 'root' with --set and
                        --cancel, and nothing with --clear.
  -s [SEC]ITEM=VALUE, --set=[SEC]ITEM=VALUE
                        A [runtime] config item and value to broadcast.
  -F FILE, --set-file=FILE, --file=FILE
                        File with config to broadcast. Can be used multiple
                        times.
  -c [SEC]ITEM, --cancel=[SEC]ITEM
                        An item-specific broadcast to cancel.
  -G FILE, --cancel-file=FILE
                        File with broadcasts to cancel. Can be used multiple
                        times.
  -C, --clear           Cancel all broadcasts, or with -p/--point,
                        -n/--namespace, cancel all broadcasts to targeted
                        namespaces and/or cycle points. Use "-C -p '*'" to
                        cancel all all-cycle broadcasts without canceling all
                        specific-cycle broadcasts.
  -e CYCLE_POINT, --expire=CYCLE_POINT
                        Cancel any broadcasts that target cycle points earlier
                        than, but not inclusive of, CYCLE_POINT.
  -d, --display         Display active broadcasts.
  -k TASKID, --display-task=TASKID
                        Print active broadcasts for a given task
                        (NAME.CYCLE_POINT).
  -b, --box             Use unicode box characters with -d, -k.
  -r, --raw             With -d/--display or -k/--display-task, write out the
                        broadcast config structure in raw Python form.
  --user=USER           Other user account name. This results in command
                        reinvocation on the remote account.
  --host=HOST           Other host name. This results in command reinvocation
                        on the remote account.
  -v, --verbose         Verbose output mode.
  --debug               Output developer information and show exception
                        tracebacks.
  --port=INT            Suite port number on the suite host. NOTE: this is
                        retrieved automatically if non-interactive ssh is
                        configured to the suite host.
  --use-ssh             Use ssh to re-invoke the command on the suite host.
  --ssh-cylc=SSH_CYLC   Location of cylc executable on remote ssh commands.
  --no-login            Do not use a login shell to run remote ssh commands.
                        The default is to use a login shell.
  --comms-timeout=SEC, --pyro-timeout=SEC
                        Set a timeout for network connections to the running
                        suite. The default is no timeout. For task messaging
                        connections see site/user config file documentation.
  --print-uuid          Print the client UUID to stderr. This can be matched
                        to information logged by the receiving suite server
                        program.
  --set-uuid=UUID       Set the client UUID manually (e.g. from prior use of
                        --print-uuid). This can be used to log multiple
                        commands under the same UUID (but note that only the
                        first [info] command from the same client ID will be
                        logged unless the suite is running in debug mode).
  -f, --force           Do not ask for confirmation before acting. Note that
                        it is not necessary to use this option if interactive
                        command prompts have been disabled in the site/user
                        config files.
\end{lstlisting}
\subsubsection{cat-log}
\label{cat-log}
\begin{lstlisting}
Usage: cylc [info] cat-log|log [OPTIONS] REG [TASK-ID] 

Print, view-in-editor, or tail-follow content, print path, or list directory,
of local or remote task job and suite server logs. Batch-system view commands
(e.g. 'qcat') are used if defined in global config and the job is running.

For standard log types use the short-cut option argument or full filename (e.g.
for job stdout "-f o" or "-f job.out" will do).

To list the local job log directory of a remote task, choose "-m l" (directory
list mode) and a local file, e.g. "-f a" (job-activity.log).

If remote job logs are retrieved to the suite host on completion (global config
'[JOB-HOST]retrieve job logs = True') and the job is not currently running, the
local (retrieved) log will be accessed unless '-o/--force-remote' is used.

Custom job logs (written to $CYLC_TASK_LOG_DIR on the job host) are available
from the GUI if listed in 'extra log files' in the suite definition. The file
name must be given here, but can be discovered with '--mode=l' (list-dir).

The correct cycle point format of the suite must be for task job logs.

Note the --host/user options are not needed to view remote job logs. They are
the general command reinvocation options for sites using ssh-based task
messaging.

Arguments:
   REG                     Suite name
   [TASK-ID]               Task ID

Options:
  -h, --help            show this help message and exit
  -f LOG, --file=LOG      Job log: a(job-activity.log), e(job.err), d(job-
                        edit.diff), j(job), o(job.out), s(job.status),
                        x(job.xtrace); default o(out).  Or <filename> for
                        custom (and standard) job logs.
  -m MODE, --mode=MODE  Mode: c(cat), e(edit), d(print-dir), l(list-dir),
                        p(print), t(tail). Default c(cat).
  -r INT, --rotation=INT
                        Suite log integer rotation number. 0 for current, 1
                        for next oldest, etc.
  -o, --force-remote    View remote logs remotely even if they have been
                        retrieved to the suite host (default False).
  -s INT, -t INT, --submit-number=INT, --try-number=INT
                        Job submit number (default=NN, i.e. latest).
  -g, --geditor         edit mode: use your configured GUI editor.
  --remote-arg=REMOTE_ARGS
                        (for internal use: continue processing on job host)
  --user=USER           Other user account name. This results in command
                        reinvocation on the remote account.
  --host=HOST           Other host name. This results in command reinvocation
                        on the remote account.
  -v, --verbose         Verbose output mode.
  --debug               Output developer information and show exception
                        tracebacks.
\end{lstlisting}
\subsubsection{cat-state}
\label{cat-state}
\begin{lstlisting}
Usage: cylc [info] cat-state [OPTIONS] REG

Print the suite state in the old state dump file format to stdout.
This command is deprecated; use "cylc ls-checkpoints" instead.

Arguments:
   REG               Suite name

Options:
  -h, --help     show this help message and exit
  -d, --dump     Use the same display format as the 'cylc dump' command.
  --user=USER    Other user account name. This results in command reinvocation
                 on the remote account.
  --host=HOST    Other host name. This results in command reinvocation on the
                 remote account.
  -v, --verbose  Verbose output mode.
  --debug        Output developer information and show exception tracebacks.
\end{lstlisting}
\subsubsection{check-software}
\label{check-software}
\begin{lstlisting}
cylc [admin] check-software [MODULES]

Check for Cylc external software dependencices, including minimum versions.

With no arguments, prints a table of results for all core & optional external
module requirements, grouped by functionality. With module argument(s),
provides an exit status for the collective result of checks on those modules.

Arguments:
    [MODULES]   Modules to include in the software check, which returns a
                zero ('pass') or non-zero ('fail') exit status, where the
                integer is equivalent to the number of modules failing. Run
                the bare check-software command to view the full list of
                valid module arguments (lower-case equivalents accepted).
\end{lstlisting}
\subsubsection{check-triggering}
\label{check-triggering}
\begin{lstlisting}
cylc [hook] check-triggering ARGS

This is a cylc shutdown event handler that compares the newly generated
suite log with a previously generated reference log "reference.log"
stored in the suite definition directory. Currently it just compares
runtime triggering information, disregarding event order and timing, and
fails the suite if there is any difference. This should be sufficient to
verify correct scheduling of any suite that is not affected by different
run-to-run conditional triggering.

1) run your suite with "cylc run --generate-reference-log" to generate
the reference log with resolved triggering information. Check manually
that the reference run was correct.
2) run reference tests with "cylc run --reference-test" - this
automatically sets the shutdown event handler along with a suite timeout
and "abort if shutdown handler fails", "abort on timeout", and "abort if
any task fails".

Reference tests can use any run mode:
 * simulation mode - tests that scheduling is equivalent to the reference
 * dummy mode - also tests that task hosting, job submission, job script
   evaluation, and cylc messaging are not broken.
 * live mode - tests everything (but takes longer with real tasks!)

 If any task fails, or if cylc itself fails, or if triggering is not
 equivalent to the reference run, the test will abort with non-zero exit
 status - so reference tests can be used as automated tests to check
 that changes to cylc have not broken your suites.
\end{lstlisting}
\subsubsection{check-versions}
\label{check-versions}
\begin{lstlisting}
Usage: cylc [discovery] check-versions [OPTIONS] SUITE 

Check the version of cylc invoked on each of SUITE's task host accounts when
CYLC_VERSION is set to *the version running this command line tool*.
Different versions are reported but are not considered an error unless the
-e|--error option is specified, because different cylc versions from 6.0.0
onward should at least be backward compatible.

It is recommended that cylc versions be installed in parallel and access
configured via the cylc version wrapper as described in the cylc INSTALL
file and User Guide. This must be done on suite and task hosts. Users then get
the latest installed version by default, or (like tasks) a particular version
if $CYLC_VERSION is defined.

Use -v/--verbose to see the command invoked to determine the remote version
(all remote cylc command invocations will be of the same form, which may be
site dependent -- see cylc global config documentation.

Arguments:
   SUITE               Suite name or path

Options:
  -h, --help            show this help message and exit
  -e, --error           Exit with error status if UNKNOWN is not available on
                        all remote accounts.
  -v, --verbose         Verbose output mode.
  --debug               Output developer information and show exception
                        tracebacks.
  --suite-owner=OWNER   Specify suite owner
  -s NAME=VALUE, --set=NAME=VALUE
                        Set the value of a Jinja2 template variable in the
                        suite definition. This option can be used multiple
                        times on the command line. NOTE: these settings
                        persist across suite restarts, but can be set again on
                        the "cylc restart" command line if they need to be
                        overridden.
  --set-file=FILE       Set the value of Jinja2 template variables in the
                        suite definition from a file containing NAME=VALUE
                        pairs (one per line). NOTE: these settings persist
                        across suite restarts, but can be set again on the
                        "cylc restart" command line if they need to be
                        overridden.
\end{lstlisting}
\subsubsection{checkpoint}
\label{checkpoint}
\begin{lstlisting}
Usage: cylc [control] checkpoint [OPTIONS] REG CHECKPOINT-NAME 

Tell suite to checkpoint its current state.


Arguments:
   REG                           Suite name
   CHECKPOINT-NAME               Checkpoint name

Options:
  -h, --help            show this help message and exit
  --user=USER           Other user account name. This results in command
                        reinvocation on the remote account.
  --host=HOST           Other host name. This results in command reinvocation
                        on the remote account.
  -v, --verbose         Verbose output mode.
  --debug               Output developer information and show exception
                        tracebacks.
  --port=INT            Suite port number on the suite host. NOTE: this is
                        retrieved automatically if non-interactive ssh is
                        configured to the suite host.
  --use-ssh             Use ssh to re-invoke the command on the suite host.
  --ssh-cylc=SSH_CYLC   Location of cylc executable on remote ssh commands.
  --no-login            Do not use a login shell to run remote ssh commands.
                        The default is to use a login shell.
  --comms-timeout=SEC, --pyro-timeout=SEC
                        Set a timeout for network connections to the running
                        suite. The default is no timeout. For task messaging
                        connections see site/user config file documentation.
  --print-uuid          Print the client UUID to stderr. This can be matched
                        to information logged by the receiving suite server
                        program.
  --set-uuid=UUID       Set the client UUID manually (e.g. from prior use of
                        --print-uuid). This can be used to log multiple
                        commands under the same UUID (but note that only the
                        first [info] command from the same client ID will be
                        logged unless the suite is running in debug mode).
  -f, --force           Do not ask for confirmation before acting. Note that
                        it is not necessary to use this option if interactive
                        command prompts have been disabled in the site/user
                        config files.
\end{lstlisting}
\subsubsection{client}
\label{client}
\begin{lstlisting}
Usage: cylc client [OPTIONS] METHOD [REG] 

(This command is for internal use.)
Invoke HTTP(S) client, expect JSON from STDIN for keyword arguments.
Use the -n option if client function requires no keyword arguments.


Arguments:
   METHOD               Network API function name
   [REG]                Suite name

Options:
  -h, --help            show this help message and exit
  -n, --no-input        Do not read from STDIN, assume null input
  --user=USER           Other user account name. This results in command
                        reinvocation on the remote account.
  --host=HOST           Other host name. This results in command reinvocation
                        on the remote account.
  -v, --verbose         Verbose output mode.
  --debug               Output developer information and show exception
                        tracebacks.
  --port=INT            Suite port number on the suite host. NOTE: this is
                        retrieved automatically if non-interactive ssh is
                        configured to the suite host.
  --use-ssh             Use ssh to re-invoke the command on the suite host.
  --ssh-cylc=SSH_CYLC   Location of cylc executable on remote ssh commands.
  --no-login            Do not use a login shell to run remote ssh commands.
                        The default is to use a login shell.
  --comms-timeout=SEC, --pyro-timeout=SEC
                        Set a timeout for network connections to the running
                        suite. The default is no timeout. For task messaging
                        connections see site/user config file documentation.
  --print-uuid          Print the client UUID to stderr. This can be matched
                        to information logged by the receiving suite server
                        program.
  --set-uuid=UUID       Set the client UUID manually (e.g. from prior use of
                        --print-uuid). This can be used to log multiple
                        commands under the same UUID (but note that only the
                        first [info] command from the same client ID will be
                        logged unless the suite is running in debug mode).
  -f, --force           Do not ask for confirmation before acting. Note that
                        it is not necessary to use this option if interactive
                        command prompts have been disabled in the site/user
                        config files.
\end{lstlisting}
\subsubsection{conditions}
\label{conditions}
\begin{lstlisting}
Usage: cylc [license] warranty [--help]
Cylc is release under the GNU General Public License v3.0
This command prints the GPL v3.0 license in full.

Options:
  --help   Print this usage message.
\end{lstlisting}
\subsubsection{cycle-point}
\label{cycle-point}
\begin{lstlisting}
Usage: cylc [util] cycle-point [OPTIONS] [POINT]

Cycle point date-time offset computation, and filename templating.

Filename templating replaces elements of a template string with corresponding
elements of the current or given cycle point.

Use ISO 8601 or posix date-time format elements:
  % cylc cyclepoint 2010080T00 --template foo-CCYY-MM-DD-Thh.nc
  foo-2010-08-08-T00.nc
  % cylc cyclepoint 2010080T00 --template foo-%Y-%m-%d-T%H.nc
  foo-2010-08-08-T00.nc

Other examples:

1) print offset from an explicit cycle point:
  % cylc [util] cycle-point --offset-hours=6 20100823T1800Z
  20100824T0000Z

2) print offset from $CYLC_TASK_CYCLE_POINT (as in suite tasks):
  % export CYLC_TASK_CYCLE_POINT=20100823T1800Z
  % cylc cycle-point --offset-hours=-6
  20100823T1200Z

3) cycle point filename templating, explicit template:
  % export CYLC_TASK_CYCLE_POINT=2010-08
  % cylc cycle-point --offset-years=2 --template=foo-CCYY-MM.nc
  foo-2012-08.nc

4) cycle point filename templating, template in a variable:
  % export CYLC_TASK_CYCLE_POINT=2010-08
  % export MYTEMPLATE=foo-CCYY-MM.nc
  % cylc cycle-point --offset-years=2 --template=MYTEMPLATE
  foo-2012-08.nc

Arguments:
   [POINT]  ISO 8601 date-time, e.g. 20140201T0000Z, default
      $CYLC_TASK_CYCLE_POINT

Options:
  -h, --help            show this help message and exit
  --offset-hours=HOURS  Add N hours to CYCLE (may be negative)
  --offset-days=DAYS    Add N days to CYCLE (N may be negative)
  --offset-months=MONTHS
                        Add N months to CYCLE (N may be negative)
  --offset-years=YEARS  Add N years to CYCLE (N may be negative)
  --offset=ISO_OFFSET   Add an ISO 8601-based interval representation to CYCLE
  --equal=POINT2        Succeed if POINT2 is equal to POINT (format agnostic).
  --template=TEMPLATE   Filename template string or variable
  --time-zone=TEMPLATE  Control the formatting of the result's timezone e.g.
                        (Z, +13:00, -hh
  --num-expanded-year-digits=NUMBER
                        Specify a number of expanded year digits to print in
                        the result
  --print-year          Print only CCYY of result
  --print-month         Print only MM of result
  --print-day           Print only DD of result
  --print-hour          Print only hh of result
\end{lstlisting}
\subsubsection{diff}
\label{diff}
\begin{lstlisting}
Usage: cylc [prep] diff|compare [OPTIONS] SUITE1 SUITE2

Compare two suite definitions and display any differences.

Differencing is done after parsing the suite.rc files so it takes
account of default values that are not explicitly defined, it disregards
the order of configuration items, and it sees any include-file content
after inlining has occurred.

Files in the suite bin directory and other sub-directories of the
suite definition directory are not currently differenced.

Arguments:
   SUITE1               Suite name or path
   SUITE2               Suite name or path

Options:
  -h, --help            show this help message and exit
  -n, --nested          print suite.rc section headings in nested form.
  --user=USER           Other user account name. This results in command
                        reinvocation on the remote account.
  --host=HOST           Other host name. This results in command reinvocation
                        on the remote account.
  -v, --verbose         Verbose output mode.
  --debug               Output developer information and show exception
                        tracebacks.
  --suite-owner=OWNER   Specify suite owner
  -s NAME=VALUE, --set=NAME=VALUE
                        Set the value of a Jinja2 template variable in the
                        suite definition. This option can be used multiple
                        times on the command line. NOTE: these settings
                        persist across suite restarts, but can be set again on
                        the "cylc restart" command line if they need to be
                        overridden.
  --set-file=FILE       Set the value of Jinja2 template variables in the
                        suite definition from a file containing NAME=VALUE
                        pairs (one per line). NOTE: these settings persist
                        across suite restarts, but can be set again on the
                        "cylc restart" command line if they need to be
                        overridden.
  --icp=CYCLE_POINT     Set initial cycle point. Required if not defined in
                        suite.rc.
\end{lstlisting}
\subsubsection{documentation}
\label{documentation}
\begin{lstlisting}
Usage: cylc [info] documentation|browse [OPTIONS] [SUITE]

View documentation in browser or PDF viewer, as per Cylc global config.

% cylc doc [OPTIONS]
   View local or internet [--www] Cylc documentation URLs.

% cylc doc [-t TASK] SUITE
    View suite or task documentation, if URLs are specified in the suite. This
parses the suite definition to extract the requested URL. Note that suite
server programs also hold suite URLs for access from the Cylc GUI.

Arguments:
   [TARGET]    File, URL, or suite name

Options:
  -h, --help            show this help message and exit
  -p, --pdf             Open the PDF User Guide directly.
  -w, --www             Open the cylc internet homepage
  -t TASK_NAME, --task=TASK_NAME
                        Browse task documentation URLs.
  -s, --stdout          Just print the URL to stdout.
  --user=USER           Other user account name. This results in command
                        reinvocation on the remote account.
  --host=HOST           Other host name. This results in command reinvocation
                        on the remote account.
  --debug               Print exception traceback on error.
  --url=URL             URL to view in your configured browser.
\end{lstlisting}
\subsubsection{dump}
\label{dump}
\begin{lstlisting}
Usage: cylc [info] dump [OPTIONS] REG 

Print state information (e.g. the state of each task) from a running
suite. For small suites 'watch cylc [info] dump SUITE' is an effective
non-GUI real time monitor (but see also 'cylc monitor').

For more information about a specific task, such as the current state of
its prerequisites and outputs, see 'cylc [info] show'.

Examples:
 Display the state of all running tasks, sorted by cycle point:
 % cylc [info] dump --tasks --sort SUITE | grep running

 Display the state of all tasks in a particular cycle point:
 % cylc [info] dump -t SUITE | grep 2010082406

Arguments:
   REG               Suite name

Options:
  -h, --help            show this help message and exit
  -g, --global          Global information only.
  -t, --tasks           Task states only.
  -r, --raw, --raw-format
                        Display raw format.
  -s, --sort            Task states only; sort by cycle point instead of name.
  --user=USER           Other user account name. This results in command
                        reinvocation on the remote account.
  --host=HOST           Other host name. This results in command reinvocation
                        on the remote account.
  -v, --verbose         Verbose output mode.
  --debug               Output developer information and show exception
                        tracebacks.
  --port=INT            Suite port number on the suite host. NOTE: this is
                        retrieved automatically if non-interactive ssh is
                        configured to the suite host.
  --use-ssh             Use ssh to re-invoke the command on the suite host.
  --ssh-cylc=SSH_CYLC   Location of cylc executable on remote ssh commands.
  --no-login            Do not use a login shell to run remote ssh commands.
                        The default is to use a login shell.
  --comms-timeout=SEC, --pyro-timeout=SEC
                        Set a timeout for network connections to the running
                        suite. The default is no timeout. For task messaging
                        connections see site/user config file documentation.
  --print-uuid          Print the client UUID to stderr. This can be matched
                        to information logged by the receiving suite server
                        program.
  --set-uuid=UUID       Set the client UUID manually (e.g. from prior use of
                        --print-uuid). This can be used to log multiple
                        commands under the same UUID (but note that only the
                        first [info] command from the same client ID will be
                        logged unless the suite is running in debug mode).
\end{lstlisting}
\subsubsection{edit}
\label{edit}
\begin{lstlisting}
Usage: cylc [prep] edit [OPTIONS] SUITE 

Edit suite definitions without having to move to their directory
locations, and with optional reversible inlining of include-files. Note
that Jinja2 suites can only be edited in raw form but the processed
version can be viewed with 'cylc [prep] view -p'.

1/cylc [prep] edit SUITE
Change to the suite definition directory and edit the suite.rc file.

2/ cylc [prep] edit -i,--inline SUITE
Edit the suite with include-files inlined between special markers. The
original suite.rc file is temporarily replaced so that the inlined
version is "live" during editing (i.e. you can run suites during
editing and cylc will pick up changes to the suite definition). The
inlined file is then split into its constituent include-files
again when you exit the editor. Include-files can be nested or
multiply-included; in the latter case only the first inclusion is
inlined (this prevents conflicting changes made to the same file).

3/ cylc [prep] edit --cleanup SUITE
Remove backup files left by previous INLINED edit sessions.

INLINED EDITING SAFETY: The suite.rc file and its include-files are
automatically backed up prior to an inlined editing session. If the
editor dies mid-session just invoke 'cylc edit -i' again to recover from
the last saved inlined file. On exiting the editor, if any of the
original include-files are found to have changed due to external
intervention during editing you will be warned and the affected files
will be written to new backups instead of overwriting the originals.
Finally, the inlined suite.rc file is also backed up on exiting
the editor, to allow recovery in case of accidental corruption of the
include-file boundary markers in the inlined file.

The edit process is spawned in the foreground as follows:
  % <editor> suite.rc
Where <editor> is defined in the cylc site/user config files.

See also 'cylc [prep] view'.

Arguments:
   SUITE               Suite name or path

Options:
  -h, --help           show this help message and exit
  -i, --inline         Edit with include-files inlined as described above.
  --cleanup            Remove backup files left by previous inlined edit
                       sessions.
  -g, --gui            Force use of the configured GUI editor.
  --user=USER          Other user account name. This results in command
                       reinvocation on the remote account.
  --host=HOST          Other host name. This results in command reinvocation
                       on the remote account.
  -v, --verbose        Verbose output mode.
  --debug              Output developer information and show exception
                       tracebacks.
  --suite-owner=OWNER  Specify suite owner
\end{lstlisting}
\subsubsection{email-suite}
\label{email-suite}
\begin{lstlisting}
Usage: cylc [hook] email-suite EVENT SUITE MESSAGE

THIS COMMAND IS OBSOLETE - use built-in email event hooks.

This is a simple suite event hook script that sends an email.
The command line arguments are supplied automatically by cylc.

For example, to get an email alert when a suite shuts down:

# SUITE.RC
[cylc]
   [[environment]]
      MAIL_ADDRESS = foo@bar.baz.waz
   [[events]]
      shutdown handler = cylc email-suite

See the Suite.rc Reference (Cylc User Guide) for more information
on suite and task event hooks and event handler scripts.
\end{lstlisting}
\subsubsection{email-task}
\label{email-task}
\begin{lstlisting}
Usage: cylc [hook] email-task EVENT SUITE TASKID MESSAGE

THIS COMMAND IS OBSOLETE - use built-in email event hooks.

A simple task event hook handler script that sends an email.
The command line arguments are supplied automatically by cylc.

For example, to get an email alert whenever any task fails:

# SUITE.RC
[cylc]
   [[environment]]
      MAIL_ADDRESS = foo@bar.baz.waz
[runtime]
   [[root]]
      [[[events]]]
         failed handler = cylc email-task

See the Suite.rc Reference (Cylc User Guide) for more information
on suite and task event hooks and event handler scripts.
\end{lstlisting}
\subsubsection{ext-trigger}
\label{ext-trigger}
\begin{lstlisting}
Usage: cylc [control] ext-trigger [OPTIONS] REG MSG ID 

Report an external event message to a suite server program. It is expected that
a task in the suite has registered the same message as an external trigger - a
special prerequisite to be satisifed by an external system, via this command,
rather than by triggering off other tasks.

The ID argument should uniquely distinguish one external trigger event from the
next. When a task's external trigger is satisfied by an incoming message, the
message ID is broadcast to all downstream tasks in the cycle point as
$CYLC_EXT_TRIGGER_ID so that they can use it - e.g. to identify a new data file
that the external triggering system is responding to.

Use the retry options in case the target suite is down or out of contact.

The suite passphrase must be installed in $HOME/.cylc/<SUITE>/.

Note: to manually trigger a task use 'cylc trigger', not this command.

Arguments:
   REG               Suite name
   MSG               External trigger message
   ID                Unique trigger ID

Options:
  -h, --help            show this help message and exit
  --max-tries=INT       Maximum number of send attempts (default 5).
  --retry-interval=SEC  Delay in seconds before retrying (default 10.0).
  --user=USER           Other user account name. This results in command
                        reinvocation on the remote account.
  --host=HOST           Other host name. This results in command reinvocation
                        on the remote account.
  -v, --verbose         Verbose output mode.
  --debug               Output developer information and show exception
                        tracebacks.
  --port=INT            Suite port number on the suite host. NOTE: this is
                        retrieved automatically if non-interactive ssh is
                        configured to the suite host.
  --use-ssh             Use ssh to re-invoke the command on the suite host.
  --ssh-cylc=SSH_CYLC   Location of cylc executable on remote ssh commands.
  --no-login            Do not use a login shell to run remote ssh commands.
                        The default is to use a login shell.
  --comms-timeout=SEC, --pyro-timeout=SEC
                        Set a timeout for network connections to the running
                        suite. The default is no timeout. For task messaging
                        connections see site/user config file documentation.
  --print-uuid          Print the client UUID to stderr. This can be matched
                        to information logged by the receiving suite server
                        program.
  --set-uuid=UUID       Set the client UUID manually (e.g. from prior use of
                        --print-uuid). This can be used to log multiple
                        commands under the same UUID (but note that only the
                        first [info] command from the same client ID will be
                        logged unless the suite is running in debug mode).
  -f, --force           Do not ask for confirmation before acting. Note that
                        it is not necessary to use this option if interactive
                        command prompts have been disabled in the site/user
                        config files.
\end{lstlisting}
\subsubsection{function-run}
\label{function-run}
\begin{lstlisting}
USAGE: cylc function-run <name> <json-args> <json-kwargs> <src-dir>

INTERNAL USE (asynchronous external trigger function execution)

Run a Python function "<name>(*args, **kwargs)" in the process pool. It must be
defined in a module of the same name. Positional and keyword arguments must be
passed in as JSON strings. <src-dir> is the suite source dir, needed to find
local xtrigger modules.
\end{lstlisting}
\subsubsection{get-directory}
\label{get-directory}
\begin{lstlisting}
Usage: cylc [prep] get-directory REG

Retrieve and print the source directory location of suite REG.
Here's an easy way to move to a suite source directory:
  $ cd $(cylc get-dir REG).

Arguments:
   SUITE               Suite name or path

Options:
  -h, --help           show this help message and exit
  --user=USER          Other user account name. This results in command
                       reinvocation on the remote account.
  --host=HOST          Other host name. This results in command reinvocation
                       on the remote account.
  -v, --verbose        Verbose output mode.
  --debug              Output developer information and show exception
                       tracebacks.
  --suite-owner=OWNER  Specify suite owner
\end{lstlisting}
\subsubsection{get-gui-config}
\label{get-gui-config}
\begin{lstlisting}
Usage: cylc [admin] get-gui-config [OPTIONS]

Print gcylc configuration settings.

By default all settings are printed. For specific sections or items
use -i/--item and wrap parent sections in square brackets:
   cylc get-gui-config --item '[themes][default]succeeded'
Multiple items can be specified at once.

Options:
  -h, --help            show this help message and exit
  -v, --verbose         Print extra information.
  --debug               Show exception tracebacks.
  -i [SEC...]ITEM, --item=[SEC...]ITEM
                        Item or section to print (multiple use allowed).
  --sparse              Only print items explicitly set in the config files.
  -p, --python          Print native Python format.
\end{lstlisting}
\subsubsection{get-host-metrics}
\label{get-host-metrics}
\begin{lstlisting}
Usage: cylc get-host-metrics [OPTIONS]

Get metrics for localhost, in the form of a JSON structure with top-level
keys as requested via the OPTIONS:

1. --load
       1, 5 and 15 minute load averages (as keys) from the 'uptime' command.
2. --memory
       Total free RAM memory, in kilobytes, from the 'free -k' command.
3. --disk-space=PATH / --disk-space=PATH1,PATH2,PATH3 (etc)
       Available disk space from the 'df -Pk' command, in kilobytes, for one
       or more valid mount directory PATHs (as listed under 'Mounted on')
       within the filesystem of localhost. Multiple PATH options can be
       specified via a comma-delimited list, each becoming a key under the
       top-level disk space key.

If no options are specified, --load and --memory are invoked by default.


Options:
  -h, --help         show this help message and exit
  -l, --load         1, 5 and 15 minute load averages from the 'uptime'
                     command.
  -m, --memory       Total memory not in use by the system, buffer or cache,
                     in KB, from '/proc/meminfo'.
  --disk-space=DISK  Available disk space, in KB, from the 'df -Pk' command.
\end{lstlisting}
\subsubsection{get-site-config}
\label{get-site-config}
\begin{lstlisting}
Usage: cylc [admin] get-site-config [OPTIONS]

Print cylc site/user configuration settings.

By default all settings are printed. For specific sections or items
use -i/--item and wrap parent sections in square brackets:
   cylc get-site-config --item '[editors]terminal'
Multiple items can be specified at once.

Options:
  -h, --help            show this help message and exit
  -i [SEC...]ITEM, --item=[SEC...]ITEM
                        Item or section to print (multiple use allowed).
  --sparse              Only print items explicitly set in the config files.
  -p, --python          Print native Python format.
  --print-run-dir       Print the configured cylc run directory.
  --print-site-dir      Print the cylc site configuration directory location.
  -v, --verbose         Print extra information.
  --debug               Show exception tracebacks.
\end{lstlisting}
\subsubsection{get-suite-config}
\label{get-suite-config}
\begin{lstlisting}
Usage: cylc [info] get-suite-config [OPTIONS] SUITE 

Print parsed suite configuration items, after runtime inheritance.

By default all settings are printed. For specific sections or items
use -i/--item and wrap sections in square brackets, e.g.:
   cylc get-suite-config --item '[scheduling]initial cycle point'
Multiple items can be retrieved at once.

By default, unset values are printed as an empty string, or (for
historical reasons) as "None" with -o/--one-line. These defaults
can be changed with the -n/--null-value option.

Example:
  |# SUITE.RC
  |[runtime]
  |    [[modelX]]
  |        [[[environment]]]
  |            FOO = foo
  |            BAR = bar

$ cylc get-suite-config --item=[runtime][modelX][environment]FOO SUITE
foo

$ cylc get-suite-config --item=[runtime][modelX][environment] SUITE
FOO = foo
BAR = bar

$ cylc get-suite-config --item=[runtime][modelX] SUITE
...
[[[environment]]]
    FOO = foo
    BAR = bar
...

Arguments:
   SUITE               Suite name or path

Options:
  -h, --help            show this help message and exit
  -i [SEC...]ITEM, --item=[SEC...]ITEM
                        Item or section to print (multiple use allowed).
  -r, --sparse          Only print items explicitly set in the config files.
  -p, --python          Print native Python format.
  -a, --all-tasks       For [runtime] items (e.g. --item='script') report
                        values for all tasks prefixed by task name.
  -n STRING, --null-value=STRING
                        The string to print for unset values (default
                        nothing).
  -m, --mark-up         Prefix each line with '!cylc!'.
  -o, --one-line        Print multiple single-value items at once.
  -t, --tasks           Print the suite task list [DEPRECATED: use 'cylc list
                        SUITE'].
  -u RUN_MODE, --run-mode=RUN_MODE
                        Get config for suite run mode.
  --user=USER           Other user account name. This results in command
                        reinvocation on the remote account.
  --host=HOST           Other host name. This results in command reinvocation
                        on the remote account.
  -v, --verbose         Verbose output mode.
  --debug               Output developer information and show exception
                        tracebacks.
  --suite-owner=OWNER   Specify suite owner
  -s NAME=VALUE, --set=NAME=VALUE
                        Set the value of a Jinja2 template variable in the
                        suite definition. This option can be used multiple
                        times on the command line. NOTE: these settings
                        persist across suite restarts, but can be set again on
                        the "cylc restart" command line if they need to be
                        overridden.
  --set-file=FILE       Set the value of Jinja2 template variables in the
                        suite definition from a file containing NAME=VALUE
                        pairs (one per line). NOTE: these settings persist
                        across suite restarts, but can be set again on the
                        "cylc restart" command line if they need to be
                        overridden.
  --icp=CYCLE_POINT     Set initial cycle point. Required if not defined in
                        suite.rc.
\end{lstlisting}
\subsubsection{get-suite-contact}
\label{get-suite-contact}
\begin{lstlisting}
Usage: cylc [info] get-suite-contact [OPTIONS] REG 

Print contact information of running suite REG.

Arguments:
   REG               Suite name

Options:
  -h, --help     show this help message and exit
  --user=USER    Other user account name. This results in command reinvocation
                 on the remote account.
  --host=HOST    Other host name. This results in command reinvocation on the
                 remote account.
  -v, --verbose  Verbose output mode.
  --debug        Output developer information and show exception tracebacks.
\end{lstlisting}
\subsubsection{get-suite-version}
\label{get-suite-version}
\begin{lstlisting}
Usage: cylc [info] get-suite-version [OPTIONS] REG 

Interrogate running suite REG to find what version of cylc is running it.

To find the version you've invoked at the command line see "cylc version".

Arguments:
   REG               Suite name

Options:
  -h, --help            show this help message and exit
  --user=USER           Other user account name. This results in command
                        reinvocation on the remote account.
  --host=HOST           Other host name. This results in command reinvocation
                        on the remote account.
  -v, --verbose         Verbose output mode.
  --debug               Output developer information and show exception
                        tracebacks.
  --port=INT            Suite port number on the suite host. NOTE: this is
                        retrieved automatically if non-interactive ssh is
                        configured to the suite host.
  --use-ssh             Use ssh to re-invoke the command on the suite host.
  --ssh-cylc=SSH_CYLC   Location of cylc executable on remote ssh commands.
  --no-login            Do not use a login shell to run remote ssh commands.
                        The default is to use a login shell.
  --comms-timeout=SEC, --pyro-timeout=SEC
                        Set a timeout for network connections to the running
                        suite. The default is no timeout. For task messaging
                        connections see site/user config file documentation.
  --print-uuid          Print the client UUID to stderr. This can be matched
                        to information logged by the receiving suite server
                        program.
  --set-uuid=UUID       Set the client UUID manually (e.g. from prior use of
                        --print-uuid). This can be used to log multiple
                        commands under the same UUID (but note that only the
                        first [info] command from the same client ID will be
                        logged unless the suite is running in debug mode).
  -f, --force           Do not ask for confirmation before acting. Note that
                        it is not necessary to use this option if interactive
                        command prompts have been disabled in the site/user
                        config files.
\end{lstlisting}
\subsubsection{gpanel}
\label{gpanel}
\begin{lstlisting}
Usage: cylc gpanel [OPTIONS]

This is a cylc scan panel applet for monitoring running suites on a set of
hosts in GNOME 2.

To install this applet, run "cylc gpanel --install" and follow the instructions
that it gives you.

This applet can be tested using the --test option.

To customize themes, copy $CYLC_DIR/etc/gcylc.rc.eg to $HOME/.cylc/gcylc.rc and
follow the instructions in the file.

To configure default suite hosts, edit "[suite servers]scan hosts" in your
global.rc file.

Options:
  -h, --help  show this help message and exit
  --compact   Switch on compact mode at runtime.
  --install   Install the panel applet.
  --test      Run in a standalone window.
\end{lstlisting}
\subsubsection{graph}
\label{graph}
\begin{lstlisting}
Usage: 1/ cylc [prep] graph [OPTIONS] SUITE [START[STOP]]
     Plot the suite.rc dependency graph for SUITE.
       2/ cylc [prep] graph [OPTIONS] -f,--file FILE
     Plot the specified dot-language graph file.
       3/ cylc [prep] graph [OPTIONS] --reference SUITE [START[STOP]]
     Print out a reference format for the dependencies in SUITE.
       4/ cylc [prep] graph [OPTIONS] --output-file FILE SUITE
     Plot SUITE dependencies to a file FILE with a extension-derived format.
     If FILE endswith ".png", output in PNG format, etc.

Plot suite dependency graphs in an interactive graph viewer.

If START is given it overrides "[visualization] initial cycle point" to
determine the start point of the graph, which defaults to the suite initial
cycle point. If STOP is given it overrides "[visualization] final cycle point"
to determine the end point of the graph, which defaults to the graph start
point plus "[visualization] number of cycle points" (which defaults to 3).
The graph start and end points are adjusted up and down to the suite initial
and final cycle points, respectively, if necessary.

The "Save" button generates an image of the current view, of format (e.g. png,
svg, jpg, eps) determined by the filename extension. If the chosen format is
not available a dialog box will show those that are available.

If the optional output filename is specified, the viewer will not open and a
graph will be written directly to the file.

GRAPH VIEWER CONTROLS:
    * Center on a node: left-click.
    * Pan view: left-drag.
    * Zoom: +/- buttons, mouse-wheel, or ctrl-left-drag.
    * Box zoom: shift-left-drag.
    * "Best Fit" and "Normal Size" buttons.
    * Left-to-right graphing mode toggle button.
    * "Ignore suicide triggers" button.
    * "Save" button: save an image of the view.
  Family (namespace) grouping controls:
    Toolbar:
    * "group" - group all families up to root.
    * "ungroup" - recursively ungroup all families.
    Right-click menu:
    * "group" - close this node's parent family.
    * "ungroup" - open this family node.
    * "recursive ungroup" - ungroup all families below this node.

Arguments:
   [SUITE]               Suite name or path
   [START]               Initial cycle point (default: suite initial point)
   [STOP]                Final cycle point (default: initial + 3 points)

Options:
  -h, --help            show this help message and exit
  -u, --ungrouped       Start with task families ungrouped (the default is
                        grouped).
  -n, --namespaces      Plot the suite namespace inheritance hierarchy (task
                        run time properties).
  -f FILE, --file=FILE  View a specific dot-language graphfile.
  --filter=NODE_NAME_PATTERN
                        Filter out one or many nodes.
  -O FILE, --output-file=FILE
                        Output to a specific file, with a format given by
                        --output-format or extrapolated from the extension.
                        '-' implies stdout in plain format.
  --output-format=FORMAT
                        Specify a format for writing out the graph to
                        --output-file e.g. png, svg, jpg, eps, dot. 'ref' is a
                        special sorted plain text format for comparison and
                        reference purposes.
  -r, --reference       Output in a sorted plain text format for comparison
                        purposes. If not given, assume --output-file=-.
  --show-suicide        Show suicide triggers.  They are not shown by default,
                        unless toggled on with the tool bar button.
  --user=USER           Other user account name. This results in command
                        reinvocation on the remote account.
  --host=HOST           Other host name. This results in command reinvocation
                        on the remote account.
  -v, --verbose         Verbose output mode.
  --debug               Output developer information and show exception
                        tracebacks.
  --suite-owner=OWNER   Specify suite owner
  -s NAME=VALUE, --set=NAME=VALUE
                        Set the value of a Jinja2 template variable in the
                        suite definition. This option can be used multiple
                        times on the command line. NOTE: these settings
                        persist across suite restarts, but can be set again on
                        the "cylc restart" command line if they need to be
                        overridden.
  --set-file=FILE       Set the value of Jinja2 template variables in the
                        suite definition from a file containing NAME=VALUE
                        pairs (one per line). NOTE: these settings persist
                        across suite restarts, but can be set again on the
                        "cylc restart" command line if they need to be
                        overridden.
\end{lstlisting}
\subsubsection{graph-diff}
\label{graph-diff}
\begin{lstlisting}
Usage: cylc graph-diff [OPTIONS] SUITE1 SUITE2 -- [GRAPH_OPTIONS_ARGS]

Difference 'cylc graph --reference' output for SUITE1 and SUITE2.

OPTIONS: Use '-g' to launch a graphical diff utility.
         Use '--diff-cmd=MY_DIFF_CMD' to use a custom diff tool.

SUITE1, SUITE2: Suite names to compare.
GRAPH_OPTIONS_ARGS: Options and arguments passed directly to cylc graph.
\end{lstlisting}
\subsubsection{gscan}
\label{gscan}
\begin{lstlisting}
Usage: cylc gscan [OPTIONS]

This is the cylc scan gui for monitoring running suites on a set of
hosts.

To customize themes copy $CYLC_DIR/etc/gcylc.rc.eg to ~/.cylc/gcylc.rc and
follow the instructions in the file.

Arguments:
   [HOSTS ...]               Hosts to scan instead of the configured hosts.

Options:
  -h, --help            show this help message and exit
  -a, --all             Scan all port ranges in known hosts.
  -n PATTERN, --name=PATTERN
                        List suites with name matching PATTERN (regular
                        expression). Defaults to any name. Can be used
                        multiple times.
  -o PATTERN, --suite-owner=PATTERN
                        List suites with owner matching PATTERN (regular
                        expression). Defaults to just your own suites. Can be
                        used multiple times.
  --comms-timeout=SEC   Set a timeout for network connections to each running
                        suite. The default is 5 seconds.
  --interval=SECONDS    Time interval (in seconds) between full updates
  --user=USER           Other user account name. This results in command
                        reinvocation on the remote account.
  --host=HOST           Other host name. This results in command reinvocation
                        on the remote account.
  -v, --verbose         Verbose output mode.
  --debug               Output developer information and show exception
                        tracebacks.
  --port=INT            Suite port number on the suite host. NOTE: this is
                        retrieved automatically if non-interactive ssh is
                        configured to the suite host.
  --use-ssh             Use ssh to re-invoke the command on the suite host.
  --ssh-cylc=SSH_CYLC   Location of cylc executable on remote ssh commands.
  --no-login            Do not use a login shell to run remote ssh commands.
                        The default is to use a login shell.
  --print-uuid          Print the client UUID to stderr. This can be matched
                        to information logged by the receiving suite server
                        program.
  --set-uuid=UUID       Set the client UUID manually (e.g. from prior use of
                        --print-uuid). This can be used to log multiple
                        commands under the same UUID (but note that only the
                        first [info] command from the same client ID will be
                        logged unless the suite is running in debug mode).
\end{lstlisting}
\subsubsection{gui}
\label{gui}
\begin{lstlisting}
Usage: cylc gui [OPTIONS] [REG] [USER_AT_HOST]
gcylc [OPTIONS] [REG] [USER_AT_HOST]

This is the cylc Graphical User Interface.

The USER_AT_HOST argument allows suite selection by 'cylc scan' output:
  cylc gui $(cylc scan | grep <suite_name>)

Local suites can be opened and switched between from within gcylc. To connect
to running remote suites (whose passphrase you have installed) you must
currently use --host and/or --user on the gcylc command line.

Available task state color themes are shown under the View menu. To customize
themes copy <cylc-dir>/etc/gcylc.rc.eg to ~/.cylc/gcylc.rc and follow the
instructions in the file.

To see current configuration settings use "cylc get-gui-config".

In the graph view, View -> Options -> "Write Graph Frames" writes .dot graph
files to the suite share directory (locally, for a remote suite). These can
be processed into a movie by $CYLC_DIR/dev/bin/live-graph-movie.sh=.

Arguments:
   [REG]                        Suite name
   [USER_AT_HOST]               user@host:port, shorthand for --user, --host & --port.

Options:
  -h, --help            show this help message and exit
  -r, --restricted      Restrict display to 'active' task states: submitted,
                        submit-failed, submit-retrying, running, failed,
                        retrying; and disable the graph view.  This may be
                        needed for very large suites. The state summary icons
                        in the status bar still represent all task proxies.
  --user=USER           Other user account name. This results in command
                        reinvocation on the remote account.
  --host=HOST           Other host name. This results in command reinvocation
                        on the remote account.
  -v, --verbose         Verbose output mode.
  --debug               Output developer information and show exception
                        tracebacks.
  --port=INT            Suite port number on the suite host. NOTE: this is
                        retrieved automatically if non-interactive ssh is
                        configured to the suite host.
  --use-ssh             Use ssh to re-invoke the command on the suite host.
  --ssh-cylc=SSH_CYLC   Location of cylc executable on remote ssh commands.
  --no-login            Do not use a login shell to run remote ssh commands.
                        The default is to use a login shell.
  --comms-timeout=SEC, --pyro-timeout=SEC
                        Set a timeout for network connections to the running
                        suite. The default is no timeout. For task messaging
                        connections see site/user config file documentation.
  --print-uuid          Print the client UUID to stderr. This can be matched
                        to information logged by the receiving suite server
                        program.
  --set-uuid=UUID       Set the client UUID manually (e.g. from prior use of
                        --print-uuid). This can be used to log multiple
                        commands under the same UUID (but note that only the
                        first [info] command from the same client ID will be
                        logged unless the suite is running in debug mode).
  -s NAME=VALUE, --set=NAME=VALUE
                        Set the value of a Jinja2 template variable in the
                        suite definition. This option can be used multiple
                        times on the command line. NOTE: these settings
                        persist across suite restarts, but can be set again on
                        the "cylc restart" command line if they need to be
                        overridden.
  --set-file=FILE       Set the value of Jinja2 template variables in the
                        suite definition from a file containing NAME=VALUE
                        pairs (one per line). NOTE: these settings persist
                        across suite restarts, but can be set again on the
                        "cylc restart" command line if they need to be
                        overridden.
\end{lstlisting}
\subsubsection{hold}
\label{hold}
\begin{lstlisting}
Usage: cylc [control] hold [OPTIONS] REG [TASKID ...] 

Hold one or more waiting tasks (cylc hold REG TASKID ...), or
a whole suite (cylc hold REG).

Held tasks do not submit even if they are ready to run.

See also 'cylc [control] release'.

TASKID is a pattern to match task proxies or task families, or groups of them:
* [CYCLE-POINT-GLOB/]TASK-NAME-GLOB[:TASK-STATE]
* [CYCLE-POINT-GLOB/]FAMILY-NAME-GLOB[:TASK-STATE]
* TASK-NAME-GLOB[.CYCLE-POINT-GLOB][:TASK-STATE]
* FAMILY-NAME-GLOB[.CYCLE-POINT-GLOB][:TASK-STATE]

For example, to match:
* all tasks in a cycle: '20200202T0000Z/*' or '*.20200202T0000Z'
* all tasks in the submitted status: ':submitted'
* retrying 'foo*' tasks in 0000Z cycles: 'foo*.*0000Z:retrying' or
  '*0000Z/foo*:retrying'
* retrying tasks in 'BAR' family: '*/BAR:retrying' or 'BAR.*:retrying'
* retrying tasks in 'BAR' or 'BAZ' families: '*/BA[RZ]:retrying' or
  'BA[RZ].*:retrying'

The old 'MATCH POINT' syntax will be automatically detected and supported. To
avoid this, use the '--no-multitask-compat' option, or use the new syntax
(with a '/' or a '.') when specifying 2 TASKID arguments.

Arguments:
   REG                        Suite name
   [TASKID ...]               Task identifiers

Options:
  -h, --help            show this help message and exit
  --after=CYCLE_POINT   Hold whole suite AFTER this cycle point.
  --user=USER           Other user account name. This results in command
                        reinvocation on the remote account.
  --host=HOST           Other host name. This results in command reinvocation
                        on the remote account.
  -v, --verbose         Verbose output mode.
  --debug               Output developer information and show exception
                        tracebacks.
  --port=INT            Suite port number on the suite host. NOTE: this is
                        retrieved automatically if non-interactive ssh is
                        configured to the suite host.
  --use-ssh             Use ssh to re-invoke the command on the suite host.
  --ssh-cylc=SSH_CYLC   Location of cylc executable on remote ssh commands.
  --no-login            Do not use a login shell to run remote ssh commands.
                        The default is to use a login shell.
  --comms-timeout=SEC, --pyro-timeout=SEC
                        Set a timeout for network connections to the running
                        suite. The default is no timeout. For task messaging
                        connections see site/user config file documentation.
  --print-uuid          Print the client UUID to stderr. This can be matched
                        to information logged by the receiving suite server
                        program.
  --set-uuid=UUID       Set the client UUID manually (e.g. from prior use of
                        --print-uuid). This can be used to log multiple
                        commands under the same UUID (but note that only the
                        first [info] command from the same client ID will be
                        logged unless the suite is running in debug mode).
  -f, --force           Do not ask for confirmation before acting. Note that
                        it is not necessary to use this option if interactive
                        command prompts have been disabled in the site/user
                        config files.
  -m, --family          (Obsolete) This option is now ignored and is retained
                        for backward compatibility only. TASKID in the
                        argument list can be used to match task and family
                        names regardless of this option.
  --no-multitask-compat
                        Disallow backward compatible multitask interface.
\end{lstlisting}
\subsubsection{import-examples}
\label{import-examples}
\begin{lstlisting}
Usage: cylc import-examples DIR

Copy the cylc example suites to DIR and register them for use under the GROUP
suite name group.

Arguments:
   DIR    destination directory
\end{lstlisting}
\subsubsection{insert}
\label{insert}
\begin{lstlisting}
Usage: cylc [control] insert [OPTIONS] REG TASKID [...] 

Insert task proxies into a running suite. Uses of insertion include:
 1) insert a task that was excluded by the suite definition at start-up.
 2) reinstate a task that was previously removed from a running suite.
 3) re-run an old task that cannot be retriggered because its task proxy
 is no longer live in the a suite.

Be aware that inserted cycling tasks keep on cycling as normal, even if
another instance of the same task exists at a later cycle (instances of
the same task at different cycles can coexist, but a newly spawned task
will not be added to the pool if it catches up to another task with the
same ID).

See also 'cylc submit', for running tasks without the scheduler.

TASKID is a pattern to match task proxies or task families, or groups of them:
* [CYCLE-POINT-GLOB/]TASK-NAME-GLOB[:TASK-STATE]
* [CYCLE-POINT-GLOB/]FAMILY-NAME-GLOB[:TASK-STATE]
* TASK-NAME-GLOB[.CYCLE-POINT-GLOB][:TASK-STATE]
* FAMILY-NAME-GLOB[.CYCLE-POINT-GLOB][:TASK-STATE]

For example, to match:
* all tasks in a cycle: '20200202T0000Z/*' or '*.20200202T0000Z'
* all tasks in the submitted status: ':submitted'
* retrying 'foo*' tasks in 0000Z cycles: 'foo*.*0000Z:retrying' or
  '*0000Z/foo*:retrying'
* retrying tasks in 'BAR' family: '*/BAR:retrying' or 'BAR.*:retrying'
* retrying tasks in 'BAR' or 'BAZ' families: '*/BA[RZ]:retrying' or
  'BA[RZ].*:retrying'

The old 'MATCH POINT' syntax will be automatically detected and supported. To
avoid this, use the '--no-multitask-compat' option, or use the new syntax
(with a '/' or a '.') when specifying 2 TASKID arguments.

Arguments:
   REG                        Suite name
   TASKID [...]               Task identifier

Options:
  -h, --help            show this help message and exit
  --stop-point=CYCLE_POINT, --remove-point=CYCLE_POINT
                        Optional hold/stop cycle point for inserted task.
  --no-check            Add task even if the provided cycle point is not valid
                        for the given task.
  --user=USER           Other user account name. This results in command
                        reinvocation on the remote account.
  --host=HOST           Other host name. This results in command reinvocation
                        on the remote account.
  -v, --verbose         Verbose output mode.
  --debug               Output developer information and show exception
                        tracebacks.
  --port=INT            Suite port number on the suite host. NOTE: this is
                        retrieved automatically if non-interactive ssh is
                        configured to the suite host.
  --use-ssh             Use ssh to re-invoke the command on the suite host.
  --ssh-cylc=SSH_CYLC   Location of cylc executable on remote ssh commands.
  --no-login            Do not use a login shell to run remote ssh commands.
                        The default is to use a login shell.
  --comms-timeout=SEC, --pyro-timeout=SEC
                        Set a timeout for network connections to the running
                        suite. The default is no timeout. For task messaging
                        connections see site/user config file documentation.
  --print-uuid          Print the client UUID to stderr. This can be matched
                        to information logged by the receiving suite server
                        program.
  --set-uuid=UUID       Set the client UUID manually (e.g. from prior use of
                        --print-uuid). This can be used to log multiple
                        commands under the same UUID (but note that only the
                        first [info] command from the same client ID will be
                        logged unless the suite is running in debug mode).
  -f, --force           Do not ask for confirmation before acting. Note that
                        it is not necessary to use this option if interactive
                        command prompts have been disabled in the site/user
                        config files.
  -m, --family          (Obsolete) This option is now ignored and is retained
                        for backward compatibility only. TASKID in the
                        argument list can be used to match task and family
                        names regardless of this option.
  --no-multitask-compat
                        Disallow backward compatible multitask interface.
\end{lstlisting}
\subsubsection{jobs-kill}
\label{jobs-kill}
\begin{lstlisting}
Usage: cylc [control] jobs-kill JOB-LOG-ROOT [JOB-LOG-DIR ...]

(This command is for internal use. Users should use "cylc kill".) Read job
status files to obtain the names of the batch systems and the job IDs in the
systems. Invoke the relevant batch system commands to ask the batch systems to
terminate the jobs.



Arguments:
   JOB-LOG-ROOT                    The log/job sub-directory for the suite
   [JOB-LOG-DIR ...]               A point/name/submit_num sub-directory

Options:
  -h, --help     show this help message and exit
  --user=USER    Other user account name. This results in command reinvocation
                 on the remote account.
  --host=HOST    Other host name. This results in command reinvocation on the
                 remote account.
  -v, --verbose  Verbose output mode.
  --debug        Output developer information and show exception tracebacks.
\end{lstlisting}
\subsubsection{jobs-poll}
\label{jobs-poll}
\begin{lstlisting}
Usage: cylc [control] jobs-poll JOB-LOG-ROOT [JOB-LOG-DIR ...]

(This command is for internal use. Users should use "cylc poll".) Read job
status files to obtain the statuses of the jobs. If necessary, Invoke the
relevant batch system commands to ask the batch systems for more statuses.



Arguments:
   JOB-LOG-ROOT                    The log/job sub-directory for the suite
   [JOB-LOG-DIR ...]               A point/name/submit_num sub-directory

Options:
  -h, --help     show this help message and exit
  --user=USER    Other user account name. This results in command reinvocation
                 on the remote account.
  --host=HOST    Other host name. This results in command reinvocation on the
                 remote account.
  -v, --verbose  Verbose output mode.
  --debug        Output developer information and show exception tracebacks.
\end{lstlisting}
\subsubsection{jobs-submit}
\label{jobs-submit}
\begin{lstlisting}
Usage: cylc [control] jobs-submit JOB-LOG-ROOT [JOB-LOG-DIR ...]

(This command is for internal use. Users should use "cylc submit".) Submit task
jobs to relevant batch systems. On a remote job host, this command reads the
job files from STDIN.



Arguments:
   JOB-LOG-ROOT                    The log/job sub-directory for the suite
   [JOB-LOG-DIR ...]               A point/name/submit_num sub-directory

Options:
  -h, --help     show this help message and exit
  --remote-mode  Is this being run on a remote job host?
  --utc-mode     (for remote mode) is the suite running in UTC mode?
  --user=USER    Other user account name. This results in command reinvocation
                 on the remote account.
  --host=HOST    Other host name. This results in command reinvocation on the
                 remote account.
  -v, --verbose  Verbose output mode.
  --debug        Output developer information and show exception tracebacks.
\end{lstlisting}
\subsubsection{jobscript}
\label{jobscript}
\begin{lstlisting}
Usage: cylc [prep] jobscript [OPTIONS] REG TASK

Generate a task job script and print it to stdout.

Here's how to capture the script in the vim editor:
  % cylc jobscript REG TASK | vim -
Emacs unfortunately cannot read from stdin:
  % cylc jobscript REG TASK > tmp.sh; emacs tmp.sh

This command wraps 'cylc [control] submit --dry-run'.
Other options (e.g. for suite host and owner) are passed
through to the submit command.

Options:
  -h, --help   Print this usage message.
  -e --edit    Open the jobscript in a CLI text editor.
  -g --gedit   Open the jobscript in a GUI text editor.
  --plain      Don't print the "Task Job Script Generated message."
 (see also 'cylc submit --help')

Arguments:
  REG          Registered suite name.
  TASK         Task ID (NAME.CYCLE_POINT)
\end{lstlisting}
\subsubsection{kill}
\label{kill}
\begin{lstlisting}
Usage: cylc [control] kill [OPTIONS] REG [TASKID ...] 

Kill jobs of active tasks and update their statuses accordingly.

To kill one or more tasks, "cylc kill REG TASKID ..."; to kill all active
tasks: "cylc kill REG".

TASKID is a pattern to match task proxies or task families, or groups of them:
* [CYCLE-POINT-GLOB/]TASK-NAME-GLOB[:TASK-STATE]
* [CYCLE-POINT-GLOB/]FAMILY-NAME-GLOB[:TASK-STATE]
* TASK-NAME-GLOB[.CYCLE-POINT-GLOB][:TASK-STATE]
* FAMILY-NAME-GLOB[.CYCLE-POINT-GLOB][:TASK-STATE]

For example, to match:
* all tasks in a cycle: '20200202T0000Z/*' or '*.20200202T0000Z'
* all tasks in the submitted status: ':submitted'
* retrying 'foo*' tasks in 0000Z cycles: 'foo*.*0000Z:retrying' or
  '*0000Z/foo*:retrying'
* retrying tasks in 'BAR' family: '*/BAR:retrying' or 'BAR.*:retrying'
* retrying tasks in 'BAR' or 'BAZ' families: '*/BA[RZ]:retrying' or
  'BA[RZ].*:retrying'

The old 'MATCH POINT' syntax will be automatically detected and supported. To
avoid this, use the '--no-multitask-compat' option, or use the new syntax
(with a '/' or a '.') when specifying 2 TASKID arguments.

Arguments:
   REG                        Suite name
   [TASKID ...]               Task identifiers

Options:
  -h, --help            show this help message and exit
  --user=USER           Other user account name. This results in command
                        reinvocation on the remote account.
  --host=HOST           Other host name. This results in command reinvocation
                        on the remote account.
  -v, --verbose         Verbose output mode.
  --debug               Output developer information and show exception
                        tracebacks.
  --port=INT            Suite port number on the suite host. NOTE: this is
                        retrieved automatically if non-interactive ssh is
                        configured to the suite host.
  --use-ssh             Use ssh to re-invoke the command on the suite host.
  --ssh-cylc=SSH_CYLC   Location of cylc executable on remote ssh commands.
  --no-login            Do not use a login shell to run remote ssh commands.
                        The default is to use a login shell.
  --comms-timeout=SEC, --pyro-timeout=SEC
                        Set a timeout for network connections to the running
                        suite. The default is no timeout. For task messaging
                        connections see site/user config file documentation.
  --print-uuid          Print the client UUID to stderr. This can be matched
                        to information logged by the receiving suite server
                        program.
  --set-uuid=UUID       Set the client UUID manually (e.g. from prior use of
                        --print-uuid). This can be used to log multiple
                        commands under the same UUID (but note that only the
                        first [info] command from the same client ID will be
                        logged unless the suite is running in debug mode).
  -f, --force           Do not ask for confirmation before acting. Note that
                        it is not necessary to use this option if interactive
                        command prompts have been disabled in the site/user
                        config files.
  -m, --family          (Obsolete) This option is now ignored and is retained
                        for backward compatibility only. TASKID in the
                        argument list can be used to match task and family
                        names regardless of this option.
  --no-multitask-compat
                        Disallow backward compatible multitask interface.
\end{lstlisting}
\subsubsection{list}
\label{list}
\begin{lstlisting}
Usage: cylc [info|prep] list|ls [OPTIONS] SUITE 

Print runtime namespace names (tasks and families), the first-parent
inheritance graph, or actual tasks for a given cycle range.

The first-parent inheritance graph determines the primary task family
groupings that are collapsible in gcylc suite views and the graph
viewer tool. To visualize the full multiple inheritance hierarchy use:
  'cylc graph -n'.

Arguments:
   SUITE               Suite name or path

Options:
  -h, --help            show this help message and exit
  -a, --all-tasks       Print all tasks, not just those used in the graph.
  -n, --all-namespaces  Print all runtime namespaces, not just tasks.
  -m, --mro             Print the linear "method resolution order" for each
                        namespace (the multiple-inheritance precedence order
                        as determined by the C3 linearization algorithm).
  -t, --tree            Print the first-parent inheritance hierarchy in tree
                        form.
  -b, --box             With -t/--tree, using unicode box characters. Your
                        terminal must be able to display unicode characters.
  -w, --with-titles     Print namespaces titles too.
  -p START[,STOP], --points=START[,STOP]
                        Print actual task IDs from the START [through STOP]
                        cycle points.
  --user=USER           Other user account name. This results in command
                        reinvocation on the remote account.
  --host=HOST           Other host name. This results in command reinvocation
                        on the remote account.
  -v, --verbose         Verbose output mode.
  --debug               Output developer information and show exception
                        tracebacks.
  --suite-owner=OWNER   Specify suite owner
  -s NAME=VALUE, --set=NAME=VALUE
                        Set the value of a Jinja2 template variable in the
                        suite definition. This option can be used multiple
                        times on the command line. NOTE: these settings
                        persist across suite restarts, but can be set again on
                        the "cylc restart" command line if they need to be
                        overridden.
  --set-file=FILE       Set the value of Jinja2 template variables in the
                        suite definition from a file containing NAME=VALUE
                        pairs (one per line). NOTE: these settings persist
                        across suite restarts, but can be set again on the
                        "cylc restart" command line if they need to be
                        overridden.
  --icp=CYCLE_POINT     Set initial cycle point. Required if not defined in
                        suite.rc.
\end{lstlisting}
\subsubsection{ls-checkpoints}
\label{ls-checkpoints}
\begin{lstlisting}
Usage: cylc [info] ls-checkpoints [OPTIONS] REG [ID ...] 

In the absence of arguments and the --all option, list checkpoint IDs, their
time and events. Otherwise, display the latest and/or the checkpoints of suite
parameters, task pool and broadcast states in the suite runtime database.


Arguments:
   REG                    Suite name
   [ID ...]               Checkpoint ID (default=latest)

Options:
  -h, --help     show this help message and exit
  -a, --all      Display data of all available checkpoints.
  --user=USER    Other user account name. This results in command reinvocation
                 on the remote account.
  --host=HOST    Other host name. This results in command reinvocation on the
                 remote account.
  -v, --verbose  Verbose output mode.
  --debug        Output developer information and show exception tracebacks.
\end{lstlisting}
\subsubsection{message}
\label{message}
\begin{lstlisting}
Usage: cylc [task] message [OPTIONS] -- [REG] [TASK-JOB] [[SEVERITY:]MESSAGE ...] 

Record task job messages.

Send task job messages to:
- The job stdout/stderr.
- The job status file, if there is one.
- The suite server program, if communication is possible.

Task jobs use this command to record and report status such as success and
failure. Applications run by task jobs can use this command to report messages
and to report registered task outputs.

Messages can be specified as arguments. A '-' indicates that the command should
read messages from STDIN. When reading from STDIN, multiple messages are
separated by empty lines. Examples:

Single message as an argument:
 % cylc message -- "${CYLC_SUITE_NAME}" "${CYLC_TASK_JOB}" 'Hello world!'

Multiple messages as arguments:
 % cylc message -- "${CYLC_SUITE_NAME}" "${CYLC_TASK_JOB}" \
        'Hello world!' 'Hi' 'WARNING:Hey!'

Multiple messages on STDIN:
 % cylc message -- "${CYLC_SUITE_NAME}" "${CYLC_TASK_JOB}" - <<'__STDIN__'
 % Hello
 % world!
 %
 % Hi
 %
 % WARNING:Hey!
 %__STDIN__

Note "${CYLC_SUITE_NAME}" and "${CYLC_TASK_JOB}" are made available in task job
environments - you do not need to write their actual values in task scripting.

Each message can be prefixed with a severity level using the syntax 'SEVERITY:
MESSAGE'.

The default message severity is INFO. The --severity=SEVERITY option can be
used to set the default severity level for all unprefixed messages.

Note: to abort a job script with a custom error message, use cylc__job_abort:
  cylc__job_abort 'message...'
(For technical reasons this is a shell function, not a cylc sub-command.)

For backward compatibility, if number of arguments is less than or equal to 2,
the command assumes the classic interface, where all arguments are messages.
Otherwise, the first 2 arguments are assumed to be the suite name and the task
job identifier.


Arguments:
   [REG]                                  Suite name
   [TASK-JOB]                             Task job identifier CYCLE/TASK_NAME/SUBMIT_NUM
   [[SEVERITY:]MESSAGE ...]               Messages

Options:
  -h, --help            show this help message and exit
  -s SEVERITY, -p SEVERITY, --severity=SEVERITY, --priority=SEVERITY
                        Set severity levels for messages that do not have one
  --user=USER           Other user account name. This results in command
                        reinvocation on the remote account.
  --host=HOST           Other host name. This results in command reinvocation
                        on the remote account.
  -v, --verbose         Verbose output mode.
  --debug               Output developer information and show exception
                        tracebacks.
  --port=INT            Suite port number on the suite host. NOTE: this is
                        retrieved automatically if non-interactive ssh is
                        configured to the suite host.
  --use-ssh             Use ssh to re-invoke the command on the suite host.
  --ssh-cylc=SSH_CYLC   Location of cylc executable on remote ssh commands.
  --no-login            Do not use a login shell to run remote ssh commands.
                        The default is to use a login shell.
  --comms-timeout=SEC, --pyro-timeout=SEC
                        Set a timeout for network connections to the running
                        suite. The default is no timeout. For task messaging
                        connections see site/user config file documentation.
  --print-uuid          Print the client UUID to stderr. This can be matched
                        to information logged by the receiving suite server
                        program.
  --set-uuid=UUID       Set the client UUID manually (e.g. from prior use of
                        --print-uuid). This can be used to log multiple
                        commands under the same UUID (but note that only the
                        first [info] command from the same client ID will be
                        logged unless the suite is running in debug mode).
  -f, --force           Do not ask for confirmation before acting. Note that
                        it is not necessary to use this option if interactive
                        command prompts have been disabled in the site/user
                        config files.
\end{lstlisting}
\subsubsection{monitor}
\label{monitor}
\begin{lstlisting}
Usage: cylc [info] monitor [OPTIONS] REG [USER_AT_HOST] 

A terminal-based live suite monitor.  Exit with 'Ctrl-C'.

The USER_AT_HOST argument allows suite selection by 'cylc scan' output:
  cylc monitor $(cylc scan | grep <suite_name>)


Arguments:
   REG                          Suite name
   [USER_AT_HOST]               user@host:port, shorthand for --user, --host & --port.

Options:
  -h, --help            show this help message and exit
  -a, --align           Align task names. Only useful for small suites.
  -r, --restricted      Restrict display to active task states. This may be
                        useful for monitoring very large suites. The state
                        summary line still reflects all task proxies.
  -s ORDER, --sort=ORDER
                        Task sort order: "definition" or "alphanumeric".The
                        default is definition order, as determined by global
                        config. (Definition order is the order that tasks
                        appear under [runtime] in the suite definition).
  -o, --once            Show a single view then exit.
  -u, --runahead        Display task proxies in the runahead pool (off by
                        default).
  -i SECONDS, --interval=SECONDS
                        Interval between suite state retrievals, in seconds
                        (default 1).
  --user=USER           Other user account name. This results in command
                        reinvocation on the remote account.
  --host=HOST           Other host name. This results in command reinvocation
                        on the remote account.
  -v, --verbose         Verbose output mode.
  --debug               Output developer information and show exception
                        tracebacks.
  --port=INT            Suite port number on the suite host. NOTE: this is
                        retrieved automatically if non-interactive ssh is
                        configured to the suite host.
  --use-ssh             Use ssh to re-invoke the command on the suite host.
  --ssh-cylc=SSH_CYLC   Location of cylc executable on remote ssh commands.
  --no-login            Do not use a login shell to run remote ssh commands.
                        The default is to use a login shell.
  --comms-timeout=SEC, --pyro-timeout=SEC
                        Set a timeout for network connections to the running
                        suite. The default is no timeout. For task messaging
                        connections see site/user config file documentation.
  --print-uuid          Print the client UUID to stderr. This can be matched
                        to information logged by the receiving suite server
                        program.
  --set-uuid=UUID       Set the client UUID manually (e.g. from prior use of
                        --print-uuid). This can be used to log multiple
                        commands under the same UUID (but note that only the
                        first [info] command from the same client ID will be
                        logged unless the suite is running in debug mode).
\end{lstlisting}
\subsubsection{nudge}
\label{nudge}
\begin{lstlisting}
Usage: cylc [control] nudge [OPTIONS] REG 

Cause the cylc task processing loop to be invoked in a running suite.

This happens automatically when the state of any task changes such that
task processing (dependency negotation etc.) is required, or if a
clock-trigger task is ready to run.

The main reason to use this command is to update the "estimated time till
completion" intervals shown in the tree-view suite control GUI, during
periods when nothing else is happening.


Arguments:
   REG               Suite name

Options:
  -h, --help            show this help message and exit
  --user=USER           Other user account name. This results in command
                        reinvocation on the remote account.
  --host=HOST           Other host name. This results in command reinvocation
                        on the remote account.
  -v, --verbose         Verbose output mode.
  --debug               Output developer information and show exception
                        tracebacks.
  --port=INT            Suite port number on the suite host. NOTE: this is
                        retrieved automatically if non-interactive ssh is
                        configured to the suite host.
  --use-ssh             Use ssh to re-invoke the command on the suite host.
  --ssh-cylc=SSH_CYLC   Location of cylc executable on remote ssh commands.
  --no-login            Do not use a login shell to run remote ssh commands.
                        The default is to use a login shell.
  --comms-timeout=SEC, --pyro-timeout=SEC
                        Set a timeout for network connections to the running
                        suite. The default is no timeout. For task messaging
                        connections see site/user config file documentation.
  --print-uuid          Print the client UUID to stderr. This can be matched
                        to information logged by the receiving suite server
                        program.
  --set-uuid=UUID       Set the client UUID manually (e.g. from prior use of
                        --print-uuid). This can be used to log multiple
                        commands under the same UUID (but note that only the
                        first [info] command from the same client ID will be
                        logged unless the suite is running in debug mode).
  -f, --force           Do not ask for confirmation before acting. Note that
                        it is not necessary to use this option if interactive
                        command prompts have been disabled in the site/user
                        config files.
\end{lstlisting}
\subsubsection{ping}
\label{ping}
\begin{lstlisting}
Usage: cylc [discovery] ping [OPTIONS] REG [TASK] 

If suite REG is running or TASK in suite REG is currently running,
exit with success status, else exit with error status.

Arguments:
   REG                  Suite name
   [TASK]               Task NAME.CYCLE_POINT

Options:
  -h, --help            show this help message and exit
  --user=USER           Other user account name. This results in command
                        reinvocation on the remote account.
  --host=HOST           Other host name. This results in command reinvocation
                        on the remote account.
  -v, --verbose         Verbose output mode.
  --debug               Output developer information and show exception
                        tracebacks.
  --port=INT            Suite port number on the suite host. NOTE: this is
                        retrieved automatically if non-interactive ssh is
                        configured to the suite host.
  --use-ssh             Use ssh to re-invoke the command on the suite host.
  --ssh-cylc=SSH_CYLC   Location of cylc executable on remote ssh commands.
  --no-login            Do not use a login shell to run remote ssh commands.
                        The default is to use a login shell.
  --comms-timeout=SEC, --pyro-timeout=SEC
                        Set a timeout for network connections to the running
                        suite. The default is no timeout. For task messaging
                        connections see site/user config file documentation.
  --print-uuid          Print the client UUID to stderr. This can be matched
                        to information logged by the receiving suite server
                        program.
  --set-uuid=UUID       Set the client UUID manually (e.g. from prior use of
                        --print-uuid). This can be used to log multiple
                        commands under the same UUID (but note that only the
                        first [info] command from the same client ID will be
                        logged unless the suite is running in debug mode).
  -f, --force           Do not ask for confirmation before acting. Note that
                        it is not necessary to use this option if interactive
                        command prompts have been disabled in the site/user
                        config files.
\end{lstlisting}
\subsubsection{poll}
\label{poll}
\begin{lstlisting}
Usage: cylc [control] poll [OPTIONS] REG [TASKID ...] 

Poll (query) task jobs to verify and update their statuses.

Use "cylc poll REG" to poll all active tasks, or "cylc poll REG TASKID" to poll
individual tasks or families, or groups of them.

TASKID is a pattern to match task proxies or task families, or groups of them:
* [CYCLE-POINT-GLOB/]TASK-NAME-GLOB[:TASK-STATE]
* [CYCLE-POINT-GLOB/]FAMILY-NAME-GLOB[:TASK-STATE]
* TASK-NAME-GLOB[.CYCLE-POINT-GLOB][:TASK-STATE]
* FAMILY-NAME-GLOB[.CYCLE-POINT-GLOB][:TASK-STATE]

For example, to match:
* all tasks in a cycle: '20200202T0000Z/*' or '*.20200202T0000Z'
* all tasks in the submitted status: ':submitted'
* retrying 'foo*' tasks in 0000Z cycles: 'foo*.*0000Z:retrying' or
  '*0000Z/foo*:retrying'
* retrying tasks in 'BAR' family: '*/BAR:retrying' or 'BAR.*:retrying'
* retrying tasks in 'BAR' or 'BAZ' families: '*/BA[RZ]:retrying' or
  'BA[RZ].*:retrying'

The old 'MATCH POINT' syntax will be automatically detected and supported. To
avoid this, use the '--no-multitask-compat' option, or use the new syntax
(with a '/' or a '.') when specifying 2 TASKID arguments.

Arguments:
   REG                        Suite name
   [TASKID ...]               Task identifiers

Options:
  -h, --help            show this help message and exit
  -s, --succeeded       Allow polling of succeeded tasks.
  --user=USER           Other user account name. This results in command
                        reinvocation on the remote account.
  --host=HOST           Other host name. This results in command reinvocation
                        on the remote account.
  -v, --verbose         Verbose output mode.
  --debug               Output developer information and show exception
                        tracebacks.
  --port=INT            Suite port number on the suite host. NOTE: this is
                        retrieved automatically if non-interactive ssh is
                        configured to the suite host.
  --use-ssh             Use ssh to re-invoke the command on the suite host.
  --ssh-cylc=SSH_CYLC   Location of cylc executable on remote ssh commands.
  --no-login            Do not use a login shell to run remote ssh commands.
                        The default is to use a login shell.
  --comms-timeout=SEC, --pyro-timeout=SEC
                        Set a timeout for network connections to the running
                        suite. The default is no timeout. For task messaging
                        connections see site/user config file documentation.
  --print-uuid          Print the client UUID to stderr. This can be matched
                        to information logged by the receiving suite server
                        program.
  --set-uuid=UUID       Set the client UUID manually (e.g. from prior use of
                        --print-uuid). This can be used to log multiple
                        commands under the same UUID (but note that only the
                        first [info] command from the same client ID will be
                        logged unless the suite is running in debug mode).
  -f, --force           Do not ask for confirmation before acting. Note that
                        it is not necessary to use this option if interactive
                        command prompts have been disabled in the site/user
                        config files.
  -m, --family          (Obsolete) This option is now ignored and is retained
                        for backward compatibility only. TASKID in the
                        argument list can be used to match task and family
                        names regardless of this option.
  --no-multitask-compat
                        Disallow backward compatible multitask interface.
\end{lstlisting}
\subsubsection{print}
\label{print}
\begin{lstlisting}
Usage: cylc [prep] print [OPTIONS] [REGEX]

Print registered (installed) suites.

Note on result filtering:
  (a) The filter patterns are Regular Expressions, not shell globs, so
the general wildcard is '.*' (match zero or more of anything), NOT '*'.
  (b) For printing purposes there is an implicit wildcard at the end of
each pattern ('foo' is the same as 'foo/*'); use the string end marker
to prevent this ('foo$' matches only literal 'foo').

Arguments:
   [REGEX]               Suite name regular expression pattern

Options:
  -h, --help     show this help message and exit
  -t, --tree     Print suites in nested tree form.
  -b, --box      Use unicode box drawing characters in tree views.
  -a, --align    Align columns.
  -x             don't print suite definition directory paths.
  -y             Don't print suite titles.
  --fail         Fail (exit 1) if no matching suites are found.
  --user=USER    Other user account name. This results in command reinvocation
                 on the remote account.
  --host=HOST    Other host name. This results in command reinvocation on the
                 remote account.
  -v, --verbose  Verbose output mode.
  --debug        Output developer information and show exception tracebacks.
\end{lstlisting}
\subsubsection{profile-battery}
\label{profile-battery}
\begin{lstlisting}
Usage: cylc profile-battery [-e [EXPERIMENT ...]] [-v [VERSION ...]]

Run profiling experiments against different versions of cylc. A list of
experiments can be specified after the -e flag, if not provided the experiment
"complex" will be chosen. A list of versions to profile against can be
specified after the -v flag, if not provided the current version will be used.

Experiments are stored in etc/profile-experiments, user experiments can be
stored in .profiling/experiments. Experiments are specified without the file
extension, experiments in .profiling/ will be chosen before those in etc/.

IMPORTANT: See etc/profile-experiments/example for an experiment template with
further details.

Versions are any valid git identifiers i.e. tags, branches, commits. To compare
results to different cylc versions either:
    * Supply cylc profile-battery with a complete list of the versions you wish
      to profile, it will then provide the option to checkout the required
      versions automatically.
    * Checkout each version manually running cylc profile-battery against only
      one version at a time. Once all results have been gathered you can then
      run cylc profile-battery with a complete list of versions.

Profiling will save results to .profiling/results.json where they can be used
for future comparisons. To list profiling results run:
    * cylc profile-battery --ls  # list all results
    * cylc profile-battery --ls -e experiment  # list all results for
                                               # experiment "experiment".
    * cylc profile-battery --ls --delete -v  6.1.2  # Delete all results for
                                                    # version 6.1.2 (prompted).

If matplotlib and numpy are installed profiling generates plots which are
saved to .profiling/plots or presented in an interactive window using the -i
flag.

Results are stored along with a checksum for the experiment file. When an
experiment file is changed previous results are maintained, future results will
be stored separately. To copy results from an older version of an experiment
into those from the current one run:
    * cylc profile-battery --promote experiment@checksum
NOTE: At present results cannot be analysed without the experiment file so old
results must be "copied" in this way to be re-used.

The results output contain only a small number of metrics, to see a full list
of results use the --full option.


Options:
  -h, --help            show this help message and exit
  -e, --experiments     Specify list of experiments to run.
  -v, --versions        Specify cylc versions to profile. Git tags, branches,
                        commits are all valid.
  -i, --interactive     Open any plots in interactive window rather saving
                        them to files.
  -p, --no-plots        Don't generate any plots.
  --ls, --list-results  List all stored results. Experiments and versions to
                        list can be specified using --experiments and
                        --versions.
  --delete              Delete stored results (to be used in combination with
                        --list-results).
  -y, --yes             Answer yes to any user input. Will check-out cylc
                        versions as required.
  --full-results, --full
                        Display all gathered metrics.
  --lobf-order=LOBF_ORDER
                        The order (int)of the line of best fit to be drawn. 0
                        for no lobf, 1 for linear, 2 for quadratic ect.
  --promote=PROMOTE     Promote results from an older version of an experiment
                        to the current version. To be used when making non-
                        functional changes to an experiment.
  --test                For development purposes, run experiment without
                        saving results and regardless of any prior runs.
\end{lstlisting}
\subsubsection{register}
\label{register}
\begin{lstlisting}
Usage: cylc [prep] register [OPTIONS] [REG] [PATH] 

Register the name REG for the suite definition in PATH. The suite server
program can then be started, stopped, and targeted by name REG. (Note that
"cylc run" can also register suites on the fly).

Registration creates a suite run directory "~/cylc-run/REG/" containing a
".service/source" symlink to the suite definition PATH. The .service directory
will also be used for server authentication files at run time.

Suite names can be hierarchical, corresponding to the path under ~/cylc-run.

  % cylc register dogs/fido PATH
Register PATH/suite.rc as dogs/fido, with run directory ~/cylc-run/dogs/fido.

  % cylc register dogs/fido
Register $PWD/suite.rc as dogs/fido.

  % cylc register
Register $PWD/suite.rc as the parent directory name: $(basename $PWD).

The same suite can be registered with multiple names; this results in multiple
suite run directories that link to the same suite definition.

To "unregister" a suite, delete or rename its run directory (renaming it under
~/cylc-run effectively re-registers the original suite with the new name).

Use of "--redirect" is required to allow an existing name (and run directory)
to be associated with a different suite definition. This is potentially
dangerous because the new suite will overwrite files in the existing run
directory. You should consider deleting or renaming an existing run directory
rather than just re-use it with another suite.

Arguments:
   [REG]                Suite name
   [PATH]               Suite definition directory (defaults to $PWD)

Options:
  -h, --help     show this help message and exit
  --redirect     Allow an existing suite name and run directory to be used
                 with another suite.
  --user=USER    Other user account name. This results in command reinvocation
                 on the remote account.
  --host=HOST    Other host name. This results in command reinvocation on the
                 remote account.
  -v, --verbose  Verbose output mode.
  --debug        Output developer information and show exception tracebacks.
\end{lstlisting}
\subsubsection{release}
\label{release}
\begin{lstlisting}
Usage: cylc [control] release|unhold [OPTIONS] REG [TASKID ...] 

Release one or more held tasks (cylc release REG TASKID)
or the whole suite (cylc release REG). Held tasks do not
submit even if they are ready to run.

See also 'cylc [control] hold'.

TASKID is a pattern to match task proxies or task families, or groups of them:
* [CYCLE-POINT-GLOB/]TASK-NAME-GLOB[:TASK-STATE]
* [CYCLE-POINT-GLOB/]FAMILY-NAME-GLOB[:TASK-STATE]
* TASK-NAME-GLOB[.CYCLE-POINT-GLOB][:TASK-STATE]
* FAMILY-NAME-GLOB[.CYCLE-POINT-GLOB][:TASK-STATE]

For example, to match:
* all tasks in a cycle: '20200202T0000Z/*' or '*.20200202T0000Z'
* all tasks in the submitted status: ':submitted'
* retrying 'foo*' tasks in 0000Z cycles: 'foo*.*0000Z:retrying' or
  '*0000Z/foo*:retrying'
* retrying tasks in 'BAR' family: '*/BAR:retrying' or 'BAR.*:retrying'
* retrying tasks in 'BAR' or 'BAZ' families: '*/BA[RZ]:retrying' or
  'BA[RZ].*:retrying'

The old 'MATCH POINT' syntax will be automatically detected and supported. To
avoid this, use the '--no-multitask-compat' option, or use the new syntax
(with a '/' or a '.') when specifying 2 TASKID arguments.

Arguments:
   REG                        Suite name
   [TASKID ...]               Task identifiers

Options:
  -h, --help            show this help message and exit
  --user=USER           Other user account name. This results in command
                        reinvocation on the remote account.
  --host=HOST           Other host name. This results in command reinvocation
                        on the remote account.
  -v, --verbose         Verbose output mode.
  --debug               Output developer information and show exception
                        tracebacks.
  --port=INT            Suite port number on the suite host. NOTE: this is
                        retrieved automatically if non-interactive ssh is
                        configured to the suite host.
  --use-ssh             Use ssh to re-invoke the command on the suite host.
  --ssh-cylc=SSH_CYLC   Location of cylc executable on remote ssh commands.
  --no-login            Do not use a login shell to run remote ssh commands.
                        The default is to use a login shell.
  --comms-timeout=SEC, --pyro-timeout=SEC
                        Set a timeout for network connections to the running
                        suite. The default is no timeout. For task messaging
                        connections see site/user config file documentation.
  --print-uuid          Print the client UUID to stderr. This can be matched
                        to information logged by the receiving suite server
                        program.
  --set-uuid=UUID       Set the client UUID manually (e.g. from prior use of
                        --print-uuid). This can be used to log multiple
                        commands under the same UUID (but note that only the
                        first [info] command from the same client ID will be
                        logged unless the suite is running in debug mode).
  -f, --force           Do not ask for confirmation before acting. Note that
                        it is not necessary to use this option if interactive
                        command prompts have been disabled in the site/user
                        config files.
  -m, --family          (Obsolete) This option is now ignored and is retained
                        for backward compatibility only. TASKID in the
                        argument list can be used to match task and family
                        names regardless of this option.
  --no-multitask-compat
                        Disallow backward compatible multitask interface.
\end{lstlisting}
\subsubsection{reload}
\label{reload}
\begin{lstlisting}
Usage: cylc [control] reload [OPTIONS] REG 

Tell a suite to reload its definition at run time. All settings
including task definitions, with the exception of suite log
configuration, can be changed on reload. Note that defined tasks can be
be added to or removed from a running suite with the 'cylc insert' and
'cylc remove' commands, without reloading. This command also allows
addition and removal of actual task definitions, and therefore insertion
of tasks that were not defined at all when the suite started (you will
still need to manually insert a particular instance of a newly defined
task). Live task proxies that are orphaned by a reload (i.e. their task
definitions have been removed) will be removed from the task pool if
they have not started running yet. Changes to task definitions take
effect immediately, unless a task is already running at reload time.

If the suite was started with Jinja2 template variables set on the
command line (cylc run --set FOO=bar REG) the same template settings
apply to the reload (only changes to the suite.rc file itself are
reloaded).

If the modified suite definition does not parse, failure to reload will
be reported but no harm will be done to the running suite.

Arguments:
   REG               Suite name

Options:
  -h, --help            show this help message and exit
  --user=USER           Other user account name. This results in command
                        reinvocation on the remote account.
  --host=HOST           Other host name. This results in command reinvocation
                        on the remote account.
  -v, --verbose         Verbose output mode.
  --debug               Output developer information and show exception
                        tracebacks.
  --port=INT            Suite port number on the suite host. NOTE: this is
                        retrieved automatically if non-interactive ssh is
                        configured to the suite host.
  --use-ssh             Use ssh to re-invoke the command on the suite host.
  --ssh-cylc=SSH_CYLC   Location of cylc executable on remote ssh commands.
  --no-login            Do not use a login shell to run remote ssh commands.
                        The default is to use a login shell.
  --comms-timeout=SEC, --pyro-timeout=SEC
                        Set a timeout for network connections to the running
                        suite. The default is no timeout. For task messaging
                        connections see site/user config file documentation.
  --print-uuid          Print the client UUID to stderr. This can be matched
                        to information logged by the receiving suite server
                        program.
  --set-uuid=UUID       Set the client UUID manually (e.g. from prior use of
                        --print-uuid). This can be used to log multiple
                        commands under the same UUID (but note that only the
                        first [info] command from the same client ID will be
                        logged unless the suite is running in debug mode).
  -f, --force           Do not ask for confirmation before acting. Note that
                        it is not necessary to use this option if interactive
                        command prompts have been disabled in the site/user
                        config files.
\end{lstlisting}
\subsubsection{remote-init}
\label{remote-init}
\begin{lstlisting}
Usage: cylc [task] remote-init [--indirect-comm=ssh] UUID RUND

(This command is for internal use.)
Install suite service files on a task remote (i.e. a [owner@]host):
    .service/contact: All task -> suite communication methods.
    .service/passphrase: Direct task -> suite HTTP(S) communication only.
    .service/ssl.cert: Direct task -> suite HTTPS communication only.

Content of items to install from a tar file read from STDIN.

Return:
    0:
        On success or if initialisation not required:
        - Print SuiteSrvFilesManager.REMOTE_INIT_NOT_REQUIRED if initialisation
          not required (e.g. remote has shared file system with suite host).
        - Print SuiteSrvFilesManager.REMOTE_INIT_DONE on success.
    1:
        On failure.



Arguments:
   UUID               UUID of current suite server process
   RUND               The run directory of the suite

Options:
  -h, --help            show this help message and exit
  --indirect-comm=METHOD
                        specify use of indirect communication via e.g. ssh
  --user=USER           Other user account name. This results in command
                        reinvocation on the remote account.
  --host=HOST           Other host name. This results in command reinvocation
                        on the remote account.
  -v, --verbose         Verbose output mode.
  --debug               Output developer information and show exception
                        tracebacks.
\end{lstlisting}
\subsubsection{remote-tidy}
\label{remote-tidy}
\begin{lstlisting}
Usage: cylc [task] remote-tidy RUND

(This command is for internal use.)
Remove ".service/contact" from a task remote (i.e. a [owner@]host).
Remove ".service" directory on the remote if emptied.



Arguments:
   RUND               The run directory of the suite

Options:
  -h, --help     show this help message and exit
  --user=USER    Other user account name. This results in command reinvocation
                 on the remote account.
  --host=HOST    Other host name. This results in command reinvocation on the
                 remote account.
  -v, --verbose  Verbose output mode.
  --debug        Output developer information and show exception tracebacks.
\end{lstlisting}
\subsubsection{remove}
\label{remove}
\begin{lstlisting}
Usage: cylc [control] remove [OPTIONS] REG TASKID [...] 

Remove one or more tasks (cylc remove REG TASKID), or all tasks with a
given cycle point (cylc remove REG *.POINT) from a running suite.

Tasks will spawn successors first if they have not done so already.

TASKID is a pattern to match task proxies or task families, or groups of them:
* [CYCLE-POINT-GLOB/]TASK-NAME-GLOB[:TASK-STATE]
* [CYCLE-POINT-GLOB/]FAMILY-NAME-GLOB[:TASK-STATE]
* TASK-NAME-GLOB[.CYCLE-POINT-GLOB][:TASK-STATE]
* FAMILY-NAME-GLOB[.CYCLE-POINT-GLOB][:TASK-STATE]

For example, to match:
* all tasks in a cycle: '20200202T0000Z/*' or '*.20200202T0000Z'
* all tasks in the submitted status: ':submitted'
* retrying 'foo*' tasks in 0000Z cycles: 'foo*.*0000Z:retrying' or
  '*0000Z/foo*:retrying'
* retrying tasks in 'BAR' family: '*/BAR:retrying' or 'BAR.*:retrying'
* retrying tasks in 'BAR' or 'BAZ' families: '*/BA[RZ]:retrying' or
  'BA[RZ].*:retrying'

The old 'MATCH POINT' syntax will be automatically detected and supported. To
avoid this, use the '--no-multitask-compat' option, or use the new syntax
(with a '/' or a '.') when specifying 2 TASKID arguments.

Arguments:
   REG                        Suite name
   TASKID [...]               Task identifiers

Options:
  -h, --help            show this help message and exit
  --no-spawn            Do not spawn successors before removal.
  --user=USER           Other user account name. This results in command
                        reinvocation on the remote account.
  --host=HOST           Other host name. This results in command reinvocation
                        on the remote account.
  -v, --verbose         Verbose output mode.
  --debug               Output developer information and show exception
                        tracebacks.
  --port=INT            Suite port number on the suite host. NOTE: this is
                        retrieved automatically if non-interactive ssh is
                        configured to the suite host.
  --use-ssh             Use ssh to re-invoke the command on the suite host.
  --ssh-cylc=SSH_CYLC   Location of cylc executable on remote ssh commands.
  --no-login            Do not use a login shell to run remote ssh commands.
                        The default is to use a login shell.
  --comms-timeout=SEC, --pyro-timeout=SEC
                        Set a timeout for network connections to the running
                        suite. The default is no timeout. For task messaging
                        connections see site/user config file documentation.
  --print-uuid          Print the client UUID to stderr. This can be matched
                        to information logged by the receiving suite server
                        program.
  --set-uuid=UUID       Set the client UUID manually (e.g. from prior use of
                        --print-uuid). This can be used to log multiple
                        commands under the same UUID (but note that only the
                        first [info] command from the same client ID will be
                        logged unless the suite is running in debug mode).
  -f, --force           Do not ask for confirmation before acting. Note that
                        it is not necessary to use this option if interactive
                        command prompts have been disabled in the site/user
                        config files.
  -m, --family          (Obsolete) This option is now ignored and is retained
                        for backward compatibility only. TASKID in the
                        argument list can be used to match task and family
                        names regardless of this option.
  --no-multitask-compat
                        Disallow backward compatible multitask interface.
\end{lstlisting}
\subsubsection{report-timings}
\label{report-timings}
\begin{lstlisting}
Usage: cylc [util] report-timings [OPTIONS] REG

Retrieve suite timing information for wait and run time performance analysis.
Raw output and summary output (in text or HTML format) are available.  Output
is sent to standard output, unless an output filename is supplied.

Summary Output (the default):
Data stratified by host and batch system that provides a statistical
summary of
    1. Queue wait time (duration between task submission and start times)
    2. Task run time (duration between start and succeed times)
    3. Total run time (duration between task submission and succeed times)
Summary tables can be output in plain text format, or HTML with embedded SVG
boxplots.  Both summary options require the Pandas library, and the HTML
summary option requires the Matplotlib library.

Raw Output:
A flat list of tabular data that provides (for each task and cycle) the
    1. Time of successful submission
    2. Time of task start
    3. Time of task successful completion
as well as information about the batch system and remote host to permit
stratification/grouping if desired by downstream processors.

Timings are shown only for succeeded tasks.

For long-running and/or large suites (i.e. for suites with many task events),
the database query to obtain the timing information may take some time.



Arguments:
   REG               Suite name

Options:
  -h, --help            show this help message and exit
  -r, --raw             Show raw timing output suitable for custom
                        diagnostics.
  -s, --summary         Show textual summary timing output for tasks.
  -w, --web-summary     Show HTML summary timing output for tasks.
  -O OUTPUT_FILENAME, --output-file=OUTPUT_FILENAME
                        Output to a specific file
  --user=USER           Other user account name. This results in command
                        reinvocation on the remote account.
  --host=HOST           Other host name. This results in command reinvocation
                        on the remote account.
  -v, --verbose         Verbose output mode.
  --debug               Output developer information and show exception
                        tracebacks.
\end{lstlisting}
\subsubsection{reset}
\label{reset}
\begin{lstlisting}
Usage: cylc [control] reset [OPTIONS] REG [TASKID ...] 

Force tasks to a specified state, and modify their prerequisites and outputs
accordingly.

Outputs are automatically updated to reflect the new task state, except for
custom message outputs - which can be manipulated directly with "--output".

Prerequisites reflect the state of other  tasks; they are not changed except
to unset them on resetting the task state to 'waiting' or earlier.

To hold and release tasks use "cylc hold" and "cylc release".
"cylc reset --state=spawn" is deprecated: use "cylc spawn" instead.


TASKID is a pattern to match task proxies or task families, or groups of them:
* [CYCLE-POINT-GLOB/]TASK-NAME-GLOB[:TASK-STATE]
* [CYCLE-POINT-GLOB/]FAMILY-NAME-GLOB[:TASK-STATE]
* TASK-NAME-GLOB[.CYCLE-POINT-GLOB][:TASK-STATE]
* FAMILY-NAME-GLOB[.CYCLE-POINT-GLOB][:TASK-STATE]

For example, to match:
* all tasks in a cycle: '20200202T0000Z/*' or '*.20200202T0000Z'
* all tasks in the submitted status: ':submitted'
* retrying 'foo*' tasks in 0000Z cycles: 'foo*.*0000Z:retrying' or
  '*0000Z/foo*:retrying'
* retrying tasks in 'BAR' family: '*/BAR:retrying' or 'BAR.*:retrying'
* retrying tasks in 'BAR' or 'BAZ' families: '*/BA[RZ]:retrying' or
  'BA[RZ].*:retrying'

The old 'MATCH POINT' syntax will be automatically detected and supported. To
avoid this, use the '--no-multitask-compat' option, or use the new syntax
(with a '/' or a '.') when specifying 2 TASKID arguments.

Arguments:
   REG                        Suite name
   [TASKID ...]               Task identifiers

Options:
  -h, --help            show this help message and exit
  -s STATE, --state=STATE
                        Reset task state to STATE, can be succeeded, waiting,
                        submitted, failed, running, submit-failed, expired
  -O OUTPUT, --output=OUTPUT
                        Find task output by message string or trigger string,
                        set complete or incomplete with !OUTPUT, '*' to set
                        all complete, '!*' to set all incomplete. Can be used
                        more than once to reset multiple task outputs.
  --user=USER           Other user account name. This results in command
                        reinvocation on the remote account.
  --host=HOST           Other host name. This results in command reinvocation
                        on the remote account.
  -v, --verbose         Verbose output mode.
  --debug               Output developer information and show exception
                        tracebacks.
  --port=INT            Suite port number on the suite host. NOTE: this is
                        retrieved automatically if non-interactive ssh is
                        configured to the suite host.
  --use-ssh             Use ssh to re-invoke the command on the suite host.
  --ssh-cylc=SSH_CYLC   Location of cylc executable on remote ssh commands.
  --no-login            Do not use a login shell to run remote ssh commands.
                        The default is to use a login shell.
  --comms-timeout=SEC, --pyro-timeout=SEC
                        Set a timeout for network connections to the running
                        suite. The default is no timeout. For task messaging
                        connections see site/user config file documentation.
  --print-uuid          Print the client UUID to stderr. This can be matched
                        to information logged by the receiving suite server
                        program.
  --set-uuid=UUID       Set the client UUID manually (e.g. from prior use of
                        --print-uuid). This can be used to log multiple
                        commands under the same UUID (but note that only the
                        first [info] command from the same client ID will be
                        logged unless the suite is running in debug mode).
  -f, --force           Do not ask for confirmation before acting. Note that
                        it is not necessary to use this option if interactive
                        command prompts have been disabled in the site/user
                        config files.
  -m, --family          (Obsolete) This option is now ignored and is retained
                        for backward compatibility only. TASKID in the
                        argument list can be used to match task and family
                        names regardless of this option.
  --no-multitask-compat
                        Disallow backward compatible multitask interface.
\end{lstlisting}
\subsubsection{restart}
\label{restart}
\begin{lstlisting}
Usage: cylc [control] restart [OPTIONS] [REG] 

Start a suite run from the previous state. To start from scratch (cold or warm
start) see the 'cylc run' command.

The scheduler runs as a daemon unless you specify --no-detach.

Tasks recorded as submitted or running are polled at start-up to determine what
happened to them while the suite was down.

Arguments:
   [REG]               Suite name

Options:
  -h, --help            show this help message and exit
  --non-daemon          (deprecated: use --no-detach)
  -n, --no-detach       Do not daemonize the suite
  -a, --no-auto-shutdown
                        Do not shut down the suite automatically when all
                        tasks have finished. This flag overrides the
                        corresponding suite config item.
  --profile             Output profiling (performance) information
  --checkpoint=CHECKPOINT-ID
                        Specify the ID of a checkpoint to restart from
  --ignore-final-cycle-point
                        Ignore the final cycle point in the suite run
                        database. If one is specified in the suite definition
                        it will be used, however.
  --ignore-initial-cycle-point
                        Ignore the initial cycle point in the suite run
                        database. If one is specified in the suite definition
                        it will be used, however.
  --until=CYCLE_POINT   Shut down after all tasks have PASSED this cycle
                        point.
  --hold                Hold (don't run tasks) immediately on starting.
  --hold-after=CYCLE_POINT
                        Hold (don't run tasks) AFTER this cycle point.
  -m STRING, --mode=STRING
                        Run mode: live, dummy, dummy-local, simulation
                        (default live).
  --reference-log       Generate a reference log for use in reference tests.
  --reference-test      Do a test run against a previously generated reference
                        log.
  --host=HOST           Specify the host on which to start-up the suite.
                        Without this set a host will be selected using the
                        'suite servers' global config.
  --user=USER           Other user account name. This results in command
                        reinvocation on the remote account.
  -v, --verbose         Verbose output mode.
  --debug               Output developer information and show exception
                        tracebacks.
  -s NAME=VALUE, --set=NAME=VALUE
                        Set the value of a Jinja2 template variable in the
                        suite definition. This option can be used multiple
                        times on the command line. NOTE: these settings
                        persist across suite restarts, but can be set again on
                        the "cylc restart" command line if they need to be
                        overridden.
  --set-file=FILE       Set the value of Jinja2 template variables in the
                        suite definition from a file containing NAME=VALUE
                        pairs (one per line). NOTE: these settings persist
                        across suite restarts, but can be set again on the
                        "cylc restart" command line if they need to be
                        overridden.
\end{lstlisting}
\subsubsection{review}
\label{review}
\begin{lstlisting}
Usage: cylc [info] review [OPTIONS] [start [PORT]] [stop] 

Start/stop ad-hoc Cylc Review web service server for browsing users' suite
logs via an HTTP interface.

With no arguments, the status of the ad-hoc web service server is printed.

For 'cylc review start', if 'PORT' is not specified, port 8080 is used.

Arguments:
   [start [PORT]]               Start ad-hoc web service server.
   [stop]                       Stop ad-hoc web service server.

Options:
  -h, --help            show this help message and exit
  -y, --non-interactive, --yes
                        Switch off interactive prompting i.e. answer yes to
                        everything (for stop only).
  -R, --service-root    Include web service name under root of URL (for start
                        only).
\end{lstlisting}
\subsubsection{run}
\label{run}
\begin{lstlisting}
Usage: cylc [control] run|start [OPTIONS] [[REG] [START_POINT] ]

Start a suite run from scratch, ignoring dependence prior to the start point.

WARNING: this will wipe out previous suite state. To restart from a previous
state, see 'cylc restart --help'.

The scheduler will run as a daemon unless you specify --no-detach.

If the suite is not already registered (by "cylc register" or a previous run)
it will be registered on the fly before start up.

% cylc run REG
  Run the suite registered with name REG.

% cylc run
  Register $PWD/suite.rc as $(basename $PWD) and run it.
 (Note REG must be given explicitly if START_POINT is on the command line.)

A "cold start" (the default) starts from the suite initial cycle point
(specified in the suite.rc or on the command line). Any dependence on tasks
prior to the suite initial cycle point is ignored.

A "warm start" (-w/--warm) starts from a given cycle point later than the suite
initial cycle point (specified in the suite.rc). Any dependence on tasks prior
to the given warm start cycle point is ignored. The suite initial cycle point
is preserved.

Arguments:
   [REG]                       Suite name
   [START_POINT]               Initial cycle point or 'now';
                               overrides the suite definition.

Options:
  -h, --help            show this help message and exit
  --non-daemon          (deprecated: use --no-detach)
  -n, --no-detach       Do not daemonize the suite
  -a, --no-auto-shutdown
                        Do not shut down the suite automatically when all
                        tasks have finished. This flag overrides the
                        corresponding suite config item.
  --profile             Output profiling (performance) information
  -w, --warm            Warm start the suite. The default is to cold start.
  --ict                 Does nothing, option for backward compatibility only
  --until=CYCLE_POINT   Shut down after all tasks have PASSED this cycle
                        point.
  --hold                Hold (don't run tasks) immediately on starting.
  --hold-after=CYCLE_POINT
                        Hold (don't run tasks) AFTER this cycle point.
  -m STRING, --mode=STRING
                        Run mode: live, dummy, dummy-local, simulation
                        (default live).
  --reference-log       Generate a reference log for use in reference tests.
  --reference-test      Do a test run against a previously generated reference
                        log.
  --host=HOST           Specify the host on which to start-up the suite.
                        Without this set a host will be selected using the
                        'suite servers' global config.
  --user=USER           Other user account name. This results in command
                        reinvocation on the remote account.
  -v, --verbose         Verbose output mode.
  --debug               Output developer information and show exception
                        tracebacks.
  -s NAME=VALUE, --set=NAME=VALUE
                        Set the value of a Jinja2 template variable in the
                        suite definition. This option can be used multiple
                        times on the command line. NOTE: these settings
                        persist across suite restarts, but can be set again on
                        the "cylc restart" command line if they need to be
                        overridden.
  --set-file=FILE       Set the value of Jinja2 template variables in the
                        suite definition from a file containing NAME=VALUE
                        pairs (one per line). NOTE: these settings persist
                        across suite restarts, but can be set again on the
                        "cylc restart" command line if they need to be
                        overridden.
\end{lstlisting}
\subsubsection{scan}
\label{scan}
\begin{lstlisting}
Usage: cylc [discovery] scan [OPTIONS] [HOSTS ...]

Print information about running suites.

By default, it will obtain a listing of running suites for the current user
from the file system, before connecting to the suites to obtain information.
Use the -o/--suite-owner option to get information of running suites for other
users.

If a list of HOSTS is specified, it will obtain a listing of running suites by
scanning all ports in the relevant range for running suites on the specified
hosts. If the -a/--all option is specified, it will use the global
configuration "[suite servers]scan hosts" setting to determine a list of hosts
to scan.

Suite passphrases are not needed to get identity information (name and owner).
Titles, descriptions, state totals, and cycle point state totals may also be
revealed publicly, depending on global and suite authentication settings. Suite
passphrases still grant full access regardless of what is revealed publicly.

WARNING: a suite suspended with Ctrl-Z will cause port scans to hang until the
connection times out (see --comms-timeout).

Arguments:
   [HOSTS ...]               Hosts to scan instead of the configured hosts.

Options:
  -h, --help            show this help message and exit
  -a, --all             Scan all port ranges in known hosts.
  -n PATTERN, --name=PATTERN
                        List suites with name matching PATTERN (regular
                        expression). Defaults to any name. Can be used
                        multiple times.
  -o PATTERN, --suite-owner=PATTERN
                        List suites with owner matching PATTERN (regular
                        expression). Defaults to current user. Use '.*' to
                        match all known users. Can be used multiple times.
  -d, --describe        Print suite metadata if available.
  -s, --state-totals    Print number of tasks in each state if available
                        (total, and by cycle point).
  -f, --full            Print all available information about each suite.
  -c, --color, --colour
                        Print task state summaries using terminal color
                        control codes.
  -b, --no-bold         Don't use any bold text in the command output.
  --print-ports         Print the port range from the global config file.
  --comms-timeout=SEC   Set a timeout for network connections to each running
                        suite. The default is 5 seconds.
  --old, --old-format   Legacy output format ("suite owner host port").
  -r, --raw, --raw-format
                        Parsable format ("suite|owner|host|property|value").
  -j, --json, --json-format
                        JSON format.
  --user=USER           Other user account name. This results in command
                        reinvocation on the remote account.
  --host=HOST           Other host name. This results in command reinvocation
                        on the remote account.
  -v, --verbose         Verbose output mode.
  --debug               Output developer information and show exception
                        tracebacks.
  --port=INT            Suite port number on the suite host. NOTE: this is
                        retrieved automatically if non-interactive ssh is
                        configured to the suite host.
  --use-ssh             Use ssh to re-invoke the command on the suite host.
  --ssh-cylc=SSH_CYLC   Location of cylc executable on remote ssh commands.
  --no-login            Do not use a login shell to run remote ssh commands.
                        The default is to use a login shell.
  --print-uuid          Print the client UUID to stderr. This can be matched
                        to information logged by the receiving suite server
                        program.
  --set-uuid=UUID       Set the client UUID manually (e.g. from prior use of
                        --print-uuid). This can be used to log multiple
                        commands under the same UUID (but note that only the
                        first [info] command from the same client ID will be
                        logged unless the suite is running in debug mode).
\end{lstlisting}
\subsubsection{scp-transfer}
\label{scp-transfer}
\begin{lstlisting}
Usage: cylc [util] scp-transfer [OPTIONS]

An scp wrapper for transferring a list of files and/or directories
at once. The source and target scp URLs can be local or remote (scp
can transfer files between two remote hosts). Passwordless ssh must
be configured appropriately.

ENVIRONMENT VARIABLE INPUTS:
$SRCE  - list of sources (files or directories) as scp URLs.
$DEST  - parallel list of targets as scp URLs.
The source and destination lists should be space-separated.

We let scp determine the validity of source and target URLs.
Target directories are created pre-copy if they don't exist.

Options:
 -v     - verbose: print scp stdout.
 --help - print this usage message.
\end{lstlisting}
\subsubsection{search}
\label{search}
\begin{lstlisting}
Usage: cylc [prep] search|grep [OPTIONS] SUITE PATTERN [PATTERN2...] 

Search for pattern matches in suite definitions and any files in the
suite bin directory. Matches are reported by line number and suite
section. An unquoted list of PATTERNs will be converted to an OR'd
pattern. Note that the order of command line arguments conforms to
normal cylc command usage (suite name first) not that of the grep
command.

Note that this command performs a text search on the suite definition,
it does not search the data structure that results from parsing the
suite definition - so it will not report implicit default settings.

For case insenstive matching use '(?i)PATTERN'.

Arguments:
   SUITE                       Suite name or path
   PATTERN                     Python-style regular expression
   [PATTERN2...]               Additional search patterns

Options:
  -h, --help           show this help message and exit
  -x                   Do not search in the suite bin directory
  --user=USER          Other user account name. This results in command
                       reinvocation on the remote account.
  --host=HOST          Other host name. This results in command reinvocation
                       on the remote account.
  -v, --verbose        Verbose output mode.
  --debug              Output developer information and show exception
                       tracebacks.
  --suite-owner=OWNER  Specify suite owner
\end{lstlisting}
\subsubsection{set-verbosity}
\label{set-verbosity}
\begin{lstlisting}
Usage: cylc [control] set-verbosity [OPTIONS] REG LEVEL 

Change the logging severity level of a running suite.  Only messages at
or above the chosen severity level will be logged; for example, if you
choose WARNING, only warnings and critical messages will be logged.

Arguments:
   REG                 Suite name
   LEVEL               INFO, WARNING, NORMAL, CRITICAL, ERROR, DEBUG

Options:
  -h, --help            show this help message and exit
  --user=USER           Other user account name. This results in command
                        reinvocation on the remote account.
  --host=HOST           Other host name. This results in command reinvocation
                        on the remote account.
  -v, --verbose         Verbose output mode.
  --debug               Output developer information and show exception
                        tracebacks.
  --port=INT            Suite port number on the suite host. NOTE: this is
                        retrieved automatically if non-interactive ssh is
                        configured to the suite host.
  --use-ssh             Use ssh to re-invoke the command on the suite host.
  --ssh-cylc=SSH_CYLC   Location of cylc executable on remote ssh commands.
  --no-login            Do not use a login shell to run remote ssh commands.
                        The default is to use a login shell.
  --comms-timeout=SEC, --pyro-timeout=SEC
                        Set a timeout for network connections to the running
                        suite. The default is no timeout. For task messaging
                        connections see site/user config file documentation.
  --print-uuid          Print the client UUID to stderr. This can be matched
                        to information logged by the receiving suite server
                        program.
  --set-uuid=UUID       Set the client UUID manually (e.g. from prior use of
                        --print-uuid). This can be used to log multiple
                        commands under the same UUID (but note that only the
                        first [info] command from the same client ID will be
                        logged unless the suite is running in debug mode).
  -f, --force           Do not ask for confirmation before acting. Note that
                        it is not necessary to use this option if interactive
                        command prompts have been disabled in the site/user
                        config files.
\end{lstlisting}
\subsubsection{show}
\label{show}
\begin{lstlisting}
Usage: cylc [info] show [OPTIONS] REG [TASKID ...] 

Interrogate a suite server program for the suite metadata; or for the metadata
of one of its tasks; or for the current state of the prerequisites, outputs,
and clock-triggering of a specific task instance.
TASKID is a pattern to match task proxies or task families, or groups of them:
* [CYCLE-POINT-GLOB/]TASK-NAME-GLOB[:TASK-STATE]
* [CYCLE-POINT-GLOB/]FAMILY-NAME-GLOB[:TASK-STATE]
* TASK-NAME-GLOB[.CYCLE-POINT-GLOB][:TASK-STATE]
* FAMILY-NAME-GLOB[.CYCLE-POINT-GLOB][:TASK-STATE]

For example, to match:
* all tasks in a cycle: '20200202T0000Z/*' or '*.20200202T0000Z'
* all tasks in the submitted status: ':submitted'
* retrying 'foo*' tasks in 0000Z cycles: 'foo*.*0000Z:retrying' or
  '*0000Z/foo*:retrying'
* retrying tasks in 'BAR' family: '*/BAR:retrying' or 'BAR.*:retrying'
* retrying tasks in 'BAR' or 'BAZ' families: '*/BA[RZ]:retrying' or
  'BA[RZ].*:retrying'

The old 'MATCH POINT' syntax will be automatically detected and supported. To
avoid this, use the '--no-multitask-compat' option, or use the new syntax
(with a '/' or a '.') when specifying 2 TASKID arguments.

Arguments:
   REG                        Suite name
   [TASKID ...]               Task names or identifiers

Options:
  -h, --help            show this help message and exit
  --list-prereqs        Print a task's pre-requisites as a list.
  --json                Print output in JSON format.
  --user=USER           Other user account name. This results in command
                        reinvocation on the remote account.
  --host=HOST           Other host name. This results in command reinvocation
                        on the remote account.
  -v, --verbose         Verbose output mode.
  --debug               Output developer information and show exception
                        tracebacks.
  --port=INT            Suite port number on the suite host. NOTE: this is
                        retrieved automatically if non-interactive ssh is
                        configured to the suite host.
  --use-ssh             Use ssh to re-invoke the command on the suite host.
  --ssh-cylc=SSH_CYLC   Location of cylc executable on remote ssh commands.
  --no-login            Do not use a login shell to run remote ssh commands.
                        The default is to use a login shell.
  --comms-timeout=SEC, --pyro-timeout=SEC
                        Set a timeout for network connections to the running
                        suite. The default is no timeout. For task messaging
                        connections see site/user config file documentation.
  --print-uuid          Print the client UUID to stderr. This can be matched
                        to information logged by the receiving suite server
                        program.
  --set-uuid=UUID       Set the client UUID manually (e.g. from prior use of
                        --print-uuid). This can be used to log multiple
                        commands under the same UUID (but note that only the
                        first [info] command from the same client ID will be
                        logged unless the suite is running in debug mode).
  -m, --family          (Obsolete) This option is now ignored and is retained
                        for backward compatibility only. TASKID in the
                        argument list can be used to match task and family
                        names regardless of this option.
  --no-multitask-compat
                        Disallow backward compatible multitask interface.
\end{lstlisting}
\subsubsection{spawn}
\label{spawn}
\begin{lstlisting}
Usage: cylc [control] spawn [OPTIONS] REG [TASKID ...] 

Force one or more task proxies to spawn successors at the next cycle point
in their sequences.  This is useful if you need to run successive instances
of a task out of order.

TASKID is a pattern to match task proxies or task families, or groups of them:
* [CYCLE-POINT-GLOB/]TASK-NAME-GLOB[:TASK-STATE]
* [CYCLE-POINT-GLOB/]FAMILY-NAME-GLOB[:TASK-STATE]
* TASK-NAME-GLOB[.CYCLE-POINT-GLOB][:TASK-STATE]
* FAMILY-NAME-GLOB[.CYCLE-POINT-GLOB][:TASK-STATE]

For example, to match:
* all tasks in a cycle: '20200202T0000Z/*' or '*.20200202T0000Z'
* all tasks in the submitted status: ':submitted'
* retrying 'foo*' tasks in 0000Z cycles: 'foo*.*0000Z:retrying' or
  '*0000Z/foo*:retrying'
* retrying tasks in 'BAR' family: '*/BAR:retrying' or 'BAR.*:retrying'
* retrying tasks in 'BAR' or 'BAZ' families: '*/BA[RZ]:retrying' or
  'BA[RZ].*:retrying'

The old 'MATCH POINT' syntax will be automatically detected and supported. To
avoid this, use the '--no-multitask-compat' option, or use the new syntax
(with a '/' or a '.') when specifying 2 TASKID arguments.

Arguments:
   REG                        Suite name
   [TASKID ...]               Task identifiers

Options:
  -h, --help            show this help message and exit
  --user=USER           Other user account name. This results in command
                        reinvocation on the remote account.
  --host=HOST           Other host name. This results in command reinvocation
                        on the remote account.
  -v, --verbose         Verbose output mode.
  --debug               Output developer information and show exception
                        tracebacks.
  --port=INT            Suite port number on the suite host. NOTE: this is
                        retrieved automatically if non-interactive ssh is
                        configured to the suite host.
  --use-ssh             Use ssh to re-invoke the command on the suite host.
  --ssh-cylc=SSH_CYLC   Location of cylc executable on remote ssh commands.
  --no-login            Do not use a login shell to run remote ssh commands.
                        The default is to use a login shell.
  --comms-timeout=SEC, --pyro-timeout=SEC
                        Set a timeout for network connections to the running
                        suite. The default is no timeout. For task messaging
                        connections see site/user config file documentation.
  --print-uuid          Print the client UUID to stderr. This can be matched
                        to information logged by the receiving suite server
                        program.
  --set-uuid=UUID       Set the client UUID manually (e.g. from prior use of
                        --print-uuid). This can be used to log multiple
                        commands under the same UUID (but note that only the
                        first [info] command from the same client ID will be
                        logged unless the suite is running in debug mode).
  -f, --force           Do not ask for confirmation before acting. Note that
                        it is not necessary to use this option if interactive
                        command prompts have been disabled in the site/user
                        config files.
  -m, --family          (Obsolete) This option is now ignored and is retained
                        for backward compatibility only. TASKID in the
                        argument list can be used to match task and family
                        names regardless of this option.
  --no-multitask-compat
                        Disallow backward compatible multitask interface.
\end{lstlisting}
\subsubsection{stop}
\label{stop}
\begin{lstlisting}
Usage: cylc [control] stop|shutdown [OPTIONS] REG [STOP] 

Tell a suite server program to shut down. In order to prevent failures going
unnoticed, suites only shut down automatically at a final cycle point if no
failed tasks are present. There are several shutdown methods:

  1. (default) stop after current active tasks finish
  2. (--now) stop immediately, orphaning current active tasks
  3. (--kill) stop after killing current active tasks
  4. (with STOP as a cycle point) stop after cycle point STOP
  5. (with STOP as a task ID) stop after task ID STOP has succeeded
  6. (--wall-clock=T) stop after time T (an ISO 8601 date-time format e.g.
     CCYYMMDDThh:mm, CCYY-MM-DDThh, etc).

Tasks that become ready after the shutdown is ordered will be submitted
immediately if the suite is restarted.  Remaining task event handlers and job
poll and kill commands, however, will be executed prior to shutdown, unless
--now is used.

This command exits immediately unless --max-polls is greater than zero, in
which case it polls to wait for suite shutdown.

Arguments:
   REG                  Suite name
   [STOP]               a/ task POINT (cycle point), or
                            b/ ISO 8601 date-time (clock time), or
                            c/ TASK (task ID).

Options:
  -h, --help            show this help message and exit
  -k, --kill            Shut down after killing currently active tasks.
  -n, --now             Shut down without waiting for active tasks to
                        complete. If this option is specified once, wait for
                        task event handler, job poll/kill to complete. If this
                        option is specified more than once, tell the suite to
                        terminate immediately.
  -w STOP, --wall-clock=STOP
                        Shut down after time STOP (ISO 8601 formatted)
  --max-polls=INT       Maximum number of polls (default 0).
  --interval=SECS       Polling interval in seconds (default 60).
  --user=USER           Other user account name. This results in command
                        reinvocation on the remote account.
  --host=HOST           Other host name. This results in command reinvocation
                        on the remote account.
  -v, --verbose         Verbose output mode.
  --debug               Output developer information and show exception
                        tracebacks.
  --port=INT            Suite port number on the suite host. NOTE: this is
                        retrieved automatically if non-interactive ssh is
                        configured to the suite host.
  --use-ssh             Use ssh to re-invoke the command on the suite host.
  --ssh-cylc=SSH_CYLC   Location of cylc executable on remote ssh commands.
  --no-login            Do not use a login shell to run remote ssh commands.
                        The default is to use a login shell.
  --comms-timeout=SEC, --pyro-timeout=SEC
                        Set a timeout for network connections to the running
                        suite. The default is no timeout. For task messaging
                        connections see site/user config file documentation.
  --print-uuid          Print the client UUID to stderr. This can be matched
                        to information logged by the receiving suite server
                        program.
  --set-uuid=UUID       Set the client UUID manually (e.g. from prior use of
                        --print-uuid). This can be used to log multiple
                        commands under the same UUID (but note that only the
                        first [info] command from the same client ID will be
                        logged unless the suite is running in debug mode).
  -f, --force           Do not ask for confirmation before acting. Note that
                        it is not necessary to use this option if interactive
                        command prompts have been disabled in the site/user
                        config files.
\end{lstlisting}
\subsubsection{submit}
\label{submit}
\begin{lstlisting}
Usage: cylc [task] submit|single [OPTIONS] REG TASK [...] 

Submit a single task to run just as it would be submitted by its suite.  Task
messaging commands will print to stdout but will not attempt to communicate
with the suite (which does not need to be running).

For tasks present in the suite graph the given cycle point is adjusted up to
the next valid cycle point for the task. For tasks defined under runtime but
not present in the graph, the given cycle point is assumed to be valid.

WARNING: do not 'cylc submit' a task that is running in its suite at the
same time - both instances will attempt to write to the same job logs.

Arguments:
   REG                      Suite name
   TASK [...]               Family or task ID (NAME.CYCLE_POINT)

Options:
  -h, --help            show this help message and exit
  -d, --dry-run         Generate the job script for the task, but don't submit
                        it.
  --user=USER           Other user account name. This results in command
                        reinvocation on the remote account.
  --host=HOST           Other host name. This results in command reinvocation
                        on the remote account.
  -v, --verbose         Verbose output mode.
  --debug               Output developer information and show exception
                        tracebacks.
  -s NAME=VALUE, --set=NAME=VALUE
                        Set the value of a Jinja2 template variable in the
                        suite definition. This option can be used multiple
                        times on the command line. NOTE: these settings
                        persist across suite restarts, but can be set again on
                        the "cylc restart" command line if they need to be
                        overridden.
  --set-file=FILE       Set the value of Jinja2 template variables in the
                        suite definition from a file containing NAME=VALUE
                        pairs (one per line). NOTE: these settings persist
                        across suite restarts, but can be set again on the
                        "cylc restart" command line if they need to be
                        overridden.
  --icp=CYCLE_POINT     Set initial cycle point. Required if not defined in
                        suite.rc.
\end{lstlisting}
\subsubsection{suite-state}
\label{suite-state}
\begin{lstlisting}
Usage: cylc suite-state REG [OPTIONS]

Print task states retrieved from a suite database; or (with --task,
--point, and --status) poll until a given task reaches a given state; or (with
--task, --point, and --message) poll until a task receives a given message.
Polling is configurable with --interval and --max-polls; for a one-off
check use --max-polls=1. The suite database does not need to exist at
the time polling commences but allocated polls are consumed waiting for
it (consider max-polls*interval as an overall timeout).

Note for non-cycling tasks --point=1 must be provided.

For your own suites the database location is determined by your
site/user config. For other suites, e.g. those owned by others, or
mirrored suite databases, use --run-dir=DIR to specify the location.

Example usages:
  cylc suite-state REG --task=TASK --point=POINT --status=STATUS
returns 0 if TASK.POINT reaches STATUS before the maximum number of
polls, otherwise returns 1.

  cylc suite-state REG --task=TASK --point=POINT --status=STATUS --offset=PT6H
adds 6 hours to the value of CYCLE for carrying out the polling operation.

  cylc suite-state REG --task=TASK --status=STATUS --task-point
uses CYLC_TASK_CYCLE_POINT environment variable as the value for the CYCLE
to poll. This is useful when you want to use cylc suite-state in a cylc task.


Arguments:
   REG               Suite name

Options:
  -h, --help            show this help message and exit
  -t TASK, --task=TASK  Specify a task to check the state of.
  -p CYCLE, --point=CYCLE
                        Specify the cycle point to check task states for.
  -T, --task-point      Use the CYLC_TASK_CYCLE_POINT environment variable as
                        the cycle point to check task states for. Shorthand
                        for --point=$CYLC_TASK_CYCLE_POINT
  --template=TEMPLATE   Remote cyclepoint template (IGNORED - this is now
                        determined automatically).
  -d DIR, --run-dir=DIR
                        The top level cylc run directory if non-standard. The
                        database should be DIR/REG/log/db. Use to interrogate
                        suites owned by others, etc.; see note above.
  -s OFFSET, --offset=OFFSET
                        Specify an offset to add to the targeted cycle point
  -S STATUS, --status=STATUS
                        Specify a particular status or triggering condition to
                        check for. Valid triggering conditions to check for
                        include: 'fail', 'finish', 'start', 'submit' and
                        'succeed'. Valid states to check for include:
                        'runahead', 'waiting', 'held', 'queued', 'expired',
                        'ready', 'submit-failed', 'submit-retrying',
                        'submitted', 'retrying', 'running', 'failed' and
                        'succeeded'.
  -O MSG, -m MSG, --output=MSG, --message=MSG
                        Check custom task output by message string or trigger
                        string.
  --max-polls=INT       Maximum number of polls (default 10).
  --interval=SECS       Polling interval in seconds (default 60).
  --user=USER           Other user account name. This results in command
                        reinvocation on the remote account.
  --host=HOST           Other host name. This results in command reinvocation
                        on the remote account.
  -v, --verbose         Verbose output mode.
  --debug               Output developer information and show exception
                        tracebacks.
\end{lstlisting}
\subsubsection{test-battery}
\label{test-battery}
\begin{lstlisting}
cd "/home/mryan/development/python/cylc-legacy/cylc"
Usage: cylc test-battery [...]

Run automated Cylc and Parsec tests, under (by default):
   /home/mryan/development/python/cylc-legacy/cylc/tests/.

Options and arguments are appended to "prove -j $NPROC -s -r ${@:-tests}".
NPROC is the number of concurrent processes to run, which defaults to the
global config "process pool size" setting.

The tests ignore normal site/user global config and instead use the file:
   /home/mryan/development/python/cylc-legacy/cylc/etc/global-tests.rc
This should specify test job hosts under the [test battery] section, plus any
other critical settings settings, including [hosts] configuration for test job
hosts (and special batchview commands like qcat if available). Additional
global config items can be added on the fly using the create_test_globalrc
shell function defined in the test_header.

Suite run directories are only cleaned up for passing tests on the suite host.

Set "export CYLC_TEST_DEBUG=true" to print failed-test stderr to the terminal.

To change the test file comparision command from "diff -u" do (for example):
   export CYLC_TEST_DIFF_CMD='xxdiff -D'

Some test suites submit jobs to the 'at' so atd must be up on the job hosts.

Commits or Pull Requests to cylc/cylc on GitHub will trigger Travis CI to run
generic (non platform-specific) tests - see /home/mryan/development/python/cylc-legacy/cylc/.travis.yml.

By default all tests are executed.  To run just a subset of them:
  * list individual tests or test directories to run on the comand line
  * list individual tests or test directories to skip in $CYLC_TEST_SKIP
  * skip all generic tests with CYLC_TEST_RUN_GENERIC=false
  * skip all platform-specific tests with CYLC_TEST_RUN_PLATFORM=false
  List specific tests relative to /home/mryan/development/python/cylc-legacy/cylc (i.e. starting with "test/").
Some platform-specific tests are automatically skipped, depending on platform.

Platform-specific tests must set "CYLC_TEST_IS_GENERIC=false" before sourcing
the test_header.

Tests requiring the sqlite3 CLI must be skipped if sqlite3 is not installed (it
is not otherwise a Cylc software prerequisite):
| if ! which sqlite3 > /dev/null; then
|     # Skip the remaining 3 tests.
|     skip 3 "sqlite3 not installed?"
|     purge_suite $SUITE_NAME
|     exit 0
| fi

Options:
  -h, --help       Print this help message and exit.
  --chunk CHUNK    Divide the test battery into chunks and run the specified
                   chunk. CHUNK takes the format 'a/b' where 'b' is the number
                   of chunks to divide the battery into and 'a' is the number
                   of the chunk to run (1 >= a >= b).

Examples:

Run the full test suite with the default options.
  cylc test-battery
Run the full test suite with 12 processes
  cylc test-battery -j 12
Run only tests under "tests/cyclers/"
  cylc test-battery tests/cyclers
Run only "tests/cyclers/16-weekly.t" in verbose mode
  cylc test-battery -v tests/cyclers/16-weekly.t
Run only tests under "tests/cyclers/", and skip 00-daily.t
  export CYLC_TEST_SKIP=tests/cyclers/00-daily.t
  cylc test-battery tests/cyclers
Run the first quarter of the test battery
  cylc test-battery --chunk '1/4'
Re-run failed tests
  cylc test-battery --state=save
  cylc test-battery --state=failed
\end{lstlisting}
\subsubsection{trigger}
\label{trigger}
\begin{lstlisting}
Usage: cylc [control] trigger [OPTIONS] REG [TASKID ...] 

Manually trigger one or more tasks. Waiting tasks will be queued (cylc internal
queues) and will submit as normal when released by the queue; queued tasks will
submit immediately even if that violates the queue limit (so you may need to
trigger a queue-limited task twice to get it to submit).

For single tasks you can use "--edit" to edit the generated job script before
it submits, to apply one-off changes. A diff between the original and edited
job script will be saved to the task job log directory.

TASKID is a pattern to match task proxies or task families, or groups of them:
* [CYCLE-POINT-GLOB/]TASK-NAME-GLOB[:TASK-STATE]
* [CYCLE-POINT-GLOB/]FAMILY-NAME-GLOB[:TASK-STATE]
* TASK-NAME-GLOB[.CYCLE-POINT-GLOB][:TASK-STATE]
* FAMILY-NAME-GLOB[.CYCLE-POINT-GLOB][:TASK-STATE]

For example, to match:
* all tasks in a cycle: '20200202T0000Z/*' or '*.20200202T0000Z'
* all tasks in the submitted status: ':submitted'
* retrying 'foo*' tasks in 0000Z cycles: 'foo*.*0000Z:retrying' or
  '*0000Z/foo*:retrying'
* retrying tasks in 'BAR' family: '*/BAR:retrying' or 'BAR.*:retrying'
* retrying tasks in 'BAR' or 'BAZ' families: '*/BA[RZ]:retrying' or
  'BA[RZ].*:retrying'

The old 'MATCH POINT' syntax will be automatically detected and supported. To
avoid this, use the '--no-multitask-compat' option, or use the new syntax
(with a '/' or a '.') when specifying 2 TASKID arguments.

Arguments:
   REG                        Suite name
   [TASKID ...]               Task identifiers

Options:
  -h, --help            show this help message and exit
  -e, --edit            Manually edit the job script before running it.
  -g, --geditor         (with --edit) force use of the configured GUI editor.
  --user=USER           Other user account name. This results in command
                        reinvocation on the remote account.
  --host=HOST           Other host name. This results in command reinvocation
                        on the remote account.
  -v, --verbose         Verbose output mode.
  --debug               Output developer information and show exception
                        tracebacks.
  --port=INT            Suite port number on the suite host. NOTE: this is
                        retrieved automatically if non-interactive ssh is
                        configured to the suite host.
  --use-ssh             Use ssh to re-invoke the command on the suite host.
  --ssh-cylc=SSH_CYLC   Location of cylc executable on remote ssh commands.
  --no-login            Do not use a login shell to run remote ssh commands.
                        The default is to use a login shell.
  --comms-timeout=SEC, --pyro-timeout=SEC
                        Set a timeout for network connections to the running
                        suite. The default is no timeout. For task messaging
                        connections see site/user config file documentation.
  --print-uuid          Print the client UUID to stderr. This can be matched
                        to information logged by the receiving suite server
                        program.
  --set-uuid=UUID       Set the client UUID manually (e.g. from prior use of
                        --print-uuid). This can be used to log multiple
                        commands under the same UUID (but note that only the
                        first [info] command from the same client ID will be
                        logged unless the suite is running in debug mode).
  -f, --force           Do not ask for confirmation before acting. Note that
                        it is not necessary to use this option if interactive
                        command prompts have been disabled in the site/user
                        config files.
  -m, --family          (Obsolete) This option is now ignored and is retained
                        for backward compatibility only. TASKID in the
                        argument list can be used to match task and family
                        names regardless of this option.
  --no-multitask-compat
                        Disallow backward compatible multitask interface.
\end{lstlisting}
\subsubsection{upgrade-run-dir}
\label{upgrade-run-dir}
\begin{lstlisting}
Usage: cylc [admin] upgrade-run-dir SUITE

For one-off conversion of a suite run directory to cylc-6 format.

Arguments:
     SUITE    suite name or run directory path

Options:
  -h, --help  show this help message and exit
\end{lstlisting}
\subsubsection{validate}
\label{validate}
\begin{lstlisting}
Usage: cylc [prep] validate [OPTIONS] SUITE 

Validate a suite definition.

If the suite definition uses include-files reported line numbers
will correspond to the inlined version seen by the parser; use
'cylc view -i,--inline SUITE' for comparison.

Arguments:
   SUITE               Suite name or path

Options:
  -h, --help            show this help message and exit
  --strict              Fail any use of unsafe or experimental features.
                        Currently this just means naked dummy tasks (tasks
                        with no corresponding runtime section) as these may
                        result from unintentional typographic errors in task
                        names.
  -o FILENAME, --output=FILENAME
                        Specify a file name to dump the processed suite.rc.
  --profile             Output profiling (performance) information
  -u RUN_MODE, --run-mode=RUN_MODE
                        Validate for run mode.
  --user=USER           Other user account name. This results in command
                        reinvocation on the remote account.
  --host=HOST           Other host name. This results in command reinvocation
                        on the remote account.
  -v, --verbose         Verbose output mode.
  --debug               Output developer information and show exception
                        tracebacks.
  --suite-owner=OWNER   Specify suite owner
  -s NAME=VALUE, --set=NAME=VALUE
                        Set the value of a Jinja2 template variable in the
                        suite definition. This option can be used multiple
                        times on the command line. NOTE: these settings
                        persist across suite restarts, but can be set again on
                        the "cylc restart" command line if they need to be
                        overridden.
  --set-file=FILE       Set the value of Jinja2 template variables in the
                        suite definition from a file containing NAME=VALUE
                        pairs (one per line). NOTE: these settings persist
                        across suite restarts, but can be set again on the
                        "cylc restart" command line if they need to be
                        overridden.
  --icp=CYCLE_POINT     Set initial cycle point. Required if not defined in
                        suite.rc.
\end{lstlisting}
\subsubsection{view}
\label{view}
\begin{lstlisting}
Usage: cylc [prep] view [OPTIONS] SUITE 

View a read-only temporary copy of suite NAME's suite.rc file, in your
editor, after optional include-file inlining and Jinja2 preprocessing.

The edit process is spawned in the foreground as follows:
  % <editor> suite.rc
Where <editor> can be set in cylc global config.

For remote host or owner, the suite will be printed to stdout unless
the '-g,--gui' flag is used to spawn a remote GUI edit session.

See also 'cylc [prep] edit'.

Arguments:
   SUITE               Suite name or path

Options:
  -h, --help            show this help message and exit
  -i, --inline          Inline include-files.
  -e, --empy            View after EmPy template processing (implies
                        '-i/--inline' as well).
  -j, --jinja2          View after Jinja2 template processing (implies
                        '-i/--inline' as well).
  -p, --process         View after all processing (EmPy, Jinja2, inlining,
                        line-continuation joining).
  -m, --mark            (With '-i') Mark inclusions in the left margin.
  -l, --label           (With '-i') Label file inclusions with the file name.
                        Line numbers will not correspond to those reported by
                        the parser.
  --single              (With '-i') Inline only the first instances of any
                        multiply-included files. Line numbers will not
                        correspond to those reported by the parser.
  -c, --cat             Concatenate continuation lines (line numbers will not
                        correspond to those reported by the parser).
  -g, --gui             Force use of the configured GUI editor.
  --stdout              Print the suite definition to stdout.
  --mark-for-edit       (With '-i') View file inclusion markers as for 'cylc
                        edit --inline'.
  --user=USER           Other user account name. This results in command
                        reinvocation on the remote account.
  --host=HOST           Other host name. This results in command reinvocation
                        on the remote account.
  -v, --verbose         Verbose output mode.
  --debug               Output developer information and show exception
                        tracebacks.
  --suite-owner=OWNER   Specify suite owner
  -s NAME=VALUE, --set=NAME=VALUE
                        Set the value of a Jinja2 template variable in the
                        suite definition. This option can be used multiple
                        times on the command line. NOTE: these settings
                        persist across suite restarts, but can be set again on
                        the "cylc restart" command line if they need to be
                        overridden.
  --set-file=FILE       Set the value of Jinja2 template variables in the
                        suite definition from a file containing NAME=VALUE
                        pairs (one per line). NOTE: these settings persist
                        across suite restarts, but can be set again on the
                        "cylc restart" command line if they need to be
                        overridden.
\end{lstlisting}
\subsubsection{warranty}
\label{warranty}
\begin{lstlisting}
Usage: cylc [license] warranty [--help]
   Cylc is released under the GNU General Public License v3.0
This command prints the GPL v3.0 disclaimer of warranty.
Options:
  --help   Print this usage message.
\end{lstlisting}
