\section{Object Oriented Programming}
\label{ObjectOrientedProgramming}

Cylc relies heavily on Object Oriented Programming (OOP) concepts,
particularly the {\em polymorphic} nature of the task proxy objects.
An absolutely minimal explanation of this follows; 
please refer to an OOP reference for more detail.

A {\bf class} is a generalisation of data type to include behaviour
(i.e.\ functions or methods) as well as state. 

%For example, a $shape$ class could define a $position$ data member to
%hold the location of a shape object, a $move()$ method that by which
%a shape object can alter its position, and a $draw()$ method that
%causes it to display itself on screen.

An {\bf object} is a more or less self contained specific instance
of a class. This is analagous to specific integer variables being 
instances of the integer data type.

A {\bf derived class} or {\bf subclass} {\em inherits} the properties
(methods and data members) of its parent class. It can also override
specific properties, or add new properties that aren't present in the
parent. Calling a particular method on an object invokes the object's
own method if one is defined, otherwise the parent class is searched,
and so on down to the root of the inheritance graph. 

%For example, we could derive a $circle$ class from $shape$, adding a
%`radius' data member and overriding the $draw()$ to get circle objects
%to display themselves as actual circles.  Because we didn't override the
%$move()$ method, calling $circle.move()$ would invoke the base class
%method, $shape.move()$. 


{\bf Polymorphism} is the ability of one type to appear as and be used
like another type.  In OOP languages with inheritance, this usually
refers to the ability to treat derived/sub-class objects as if they were
members of a common base class. In particular, a group of mixed-type
objects can all be treated as members of a common base class. 
%For example, a group of %$circles$, $triangles$, and $squares$ could 
%be manipulated by code designed entirely to handel $shapes$; calling
%$[shape].draw()$ will invoke the right derived class $draw()$ method. 
This is a powerful mechanism because it allows existing old code,
without modification, to manipulate new objects so long as they 
derive from the original base class.
%If we later derive an entirely new kind of shape ($hexagon$, say) with
%it's own unique behaviour, the existing program, without modification,
%will process the new objects in the proper hexagon-specific way.  

In cylc, all task proxy objects are derived from a base class that 
embodies the properties and behaviour common to all task proxies. 
The scheduling algorithm works with instances of the base class so that
any current or future derived task object can be handled by the program
without modification (other than deriving the new subclass itself).

\subsection{Single- or Multi-Threaded Pyro?}
\label{Single-orMulti-ThreadedPyro?}

In single threaded mode Pyro's \lstinline=handleRequests()= returns
after at least one request (i.e.\ remote method call) was
handled, or after a timeout. Using \lstinline|timeout = None| 
allows us to process tasks only when remote method invocations
come in.  Further, we can detect the remote calls that actually change
task states, and thereby drop into the task processing code only when
necessary, which eliminates a lot of extraneous output when debugging
the task processing loop (e.g.\ in dummy mode there are a lot of remote
calls on the dummy clock object, which does not alter tasks at all). 

In multithreaded mode, \lstinline=handleRequests()= returns immediately
after creating a new request handling thread for a single remote object,
and thereafter remote method calls on that object come in asynchronously
in the dedicated thread. This is not good for cylc's scheduling
algorithm because tasks are only set running in the task processing
block which can be delayed while \lstinline=handleRequests()= blocks waiting
for a new connection to be established, even as messages that warrant
task processing are coming in on existing connections. The only way
around this seems to be to do task processing on \lstinline=handleRequests()=
timeouts which results in a lot of unnecessary processing when nothing
important is happening.

Addendum, we now use a timeout on \lstinline=handleRequests()= because
contact tasks can trigger purely on the wall clock, so we delaying task
processing when no messages are coming in may prevent these contact
tasks from triggering.   So\dots we may want to revist Multithreading\dots


