%\documentclass[11pt,a4paper]{report}
\documentclass[11pt,a4paper]{article}

\usepackage{listings}
\usepackage{amsmath}

\lstset{language=Python}

\title{Connecting Sequenz To Real Tasks}
\author{Hilary Oliver, NIWA}

\begin{document}

\maketitle
\tableofcontents

\section{Overview}

First, divide your system into a set of interdependent {\em
tasks}\footnote{A {\em task} is a single schedulable unit such as a
science model or top-level post-processing script), and specify the
pre-requisites and post-requisites for each task. Then:

\begin{itemize}
    \item write a simple {\em task definition file} for each task: 
    this specifies properties such as task name, hours the task should
    run at each day, external task launch script, and pre- and
    post-requisites.
    
    \item run {\em task-definitions/generate_task_classes.py}: this
    parses your task definition files and generates python source 
    code to define the task classes ({\em task_classes.py}). 

    \item if necessary, write the real external task launch scripts:
    these may be trivial wrappers around an existing script or
    program that runs the external task,  or sequenz may be able to call 
    existing scripts directly.

    \item finally, ensure that each external task communicates with
    sequenz at appropriate points: at the least any defined
    postrequisites must be reported when achieved, but arbitrary
    non-postrequisite messages can be sent too for logging and progress
    monitoring purposes.

\end{itemize}

\end{document}
